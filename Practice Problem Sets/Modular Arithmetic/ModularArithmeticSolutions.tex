
\documentclass[letterpaper,12pt]{article}
\newcommand{\myname}{Practice package} 
\newcommand{\mynumber}{Practice package} 
\newcommand{\hw}{1} 
\usepackage{amsmath, amsfonts, amssymb, amsthm}

\usepackage[paper=letterpaper,left=25mm,right=25mm,top=3cm,bottom=25mm]{geometry}
\usepackage{ragged2e}

\usepackage{fancyhdr} %% for details on how this work, search-engine ``fancyhdr documentation''
\pagestyle{fancy}

\lhead{Mathematics 220} % course name as top-left
\chead{Modular Arithmetic} % homework number in top-centre
\rhead{ \myname \\ \mynumber }

\lfoot{\myname} % name on bottom-left
\cfoot{Page \thepage} % page in middle
\rfoot{\mynumber} % student number on bottom-right

\renewcommand{\headrulewidth}{0.4pt}
\renewcommand{\footrulewidth}{0.4pt}

\newcommand{\set}[1]{\left\{ #1 \right\}}
%% We also redfine the negation symbol:
\renewcommand{\neg}{\sim}
\newtheorem{lemma}{Lemma}[section]
\begin{document}
\subsection*{Practice questions:}
\begin{enumerate}
\item Draw out the multiplication table for integers mod $6$. \par \centering
\begin{tabular}{|c|c|c|c|c|c|c|}
\hline
   $\cdot$  & $[0]$ & $[1]$ & $[2]$ & $[3]$ & $[4]$ & $[5]$  \\
   \hline
   $[0]$  & $[0]$  & $[0]$  & $[0]$  & $[0]$  & $[0]$  & $[0]$  \\
   \hline
   $[1]$ & $[0]$ & $[1]$ & $[2]$ & $[3]$ & $[4]$ & $[5]$ \\
   \hline
   $[2]$ & $[0]$ & $[2]$ & $[4]$ & $[0]$ & $[2]$ & $[4]$ \\
   \hline
   $[3]$ & $[0]$ & $[3]$ & $[0]$ & $[3]$ & $[0]$ & $[3]$ \\
   \hline
   $[4]$ & $[0]$ & $[4]$ & $[2]$ & $[0]$ & $[4]$ & $[2]$ \\
   \hline
   $[5]$ & $[0]$ & $[5]$ & $[4]$ & $[3]$ & $[2]$ & $[1]$ \\
   \hline
\end{tabular}
\item \justifying Let $p$ be a prime number and assume $a,b \in \mathbb{Z}$. Prove that if $p \mid ab$, then $p \mid a$ or $p \mid b$.
\begin{proof}
     Let $p$ be a prime number, $a,b \in \mathbb{Z}$, and assume $p \mid ab$. We proceed with proof by cases. \begin{itemize}
         \item If $p \mid a$, then we are done.
         \item If $p \nmid a$, since $p$ is a prime, it follows that $\gcd (a,p) = 1$. By Bézout's identity, we know there exists $x,y \in \mathbb{Z}$ such that $ax + py = 1$. Multiplying both sides by $b$ gives $abx + bpy =b$ and by assumption, we know $ab = pk$ for some $k \in \mathbb{Z}$ so $p(kx + by) = b $. Since $kx + b \in \mathbb{Z}$, it follows that $p \mid b$ as required.
     \end{itemize}
     This proves both cases and hence the result follows.
\end{proof}
\item Prove that for all $n \in \mathbb{Z}$, $n$ and $n+1$ are coprime.
\begin{proof}
    Let $n \in \mathbb{Z}$ and assume $p \mid n$ and $p \mid n+1$ for some $p \in \mathbb{Z}$. Then, $n = pk$ and $n+1 = p\ell$ for some $k, \ell \in \mathbb{Z}$. Hence, $1= n+1 - n = p(\ell + k)$ so $p \mid 1$. It follows that $p = \pm 1$ so the only natural number that can divide $n$ and $n+1$ is $1$, $n$ and $n+1$ are coprime as required.
\end{proof}
\item Let $m,n \in \mathbb{Z}$, and let $r$ be the remainder of $m$ under division by $n$. Prove that $\gcd (m,n) = \gcd (n,r)$. \begin{proof}
     Let $m,n \in \mathbb{Z}$, and let $r$ be the remainder of $m$ under division by $n$. We show that $m$ and $n$ have the same set of common divisors as $n$ and $r$. Suppose $d \mid m$ and $d \mid n$ for some $d \in \mathbb{Z}$. Notice $m \equiv 0 \mod d$ and $n \equiv 0 \mod d$, and since $m = nk + r$ for some $k \in \mathbb{Z}$, \begin{align*}
         0 \equiv m \equiv nk+ r \equiv r \mod d
     \end{align*}
So $d \mid r$ and so we have $d \mid m$, $d \mid n$, and $d \mid r$. Now suppose $e \mid n$ and $e \mid r$ for some $e \in \mathbb{Z}$. Then, $m = nk + r$, so obviously $e\mid m$. It follows that $e \mid m$, $e \mid n$ and $e \mid r$ and thus any common divisor of $m$ and $n$ is also a common divisor of $n$ and $r$. It follows that $m,n$ and $n,r$ have the same set of common divisors so they must also have the same greatest common divisors. Hence, $\gcd(m,n) = \gcd(n,r)$ as required.
\end{proof}
    \item Prove that for all odd $a,b,c \in \mathbb{Z}$, there exists no rational solutions to $ax^2 + bx + c = 0$. (Hint: Proof by contradiction under mod 2) \begin{proof}
    Let $a,b,c \in \mathbb{Z}$ and assume there exists a rational $x$ such that $ax^2 + bx + c = 0$. Then, $x = \frac{p}{q}$ for some coprime $p,q \in \mathbb{Z}$ where $q \neq 0$. Substituting $x$ for $\frac{p}{q}$ and simplifying the equation gives $ap^2 + bpq + cq^2 = 0$. Consider this equation under mod 2. \begin{itemize}
        \item $p,q$ cannot both be even by coprimeness.
        \item If $p$ is even and $q$ is odd, then \begin{align*}
            ap^2 + bpq + cq^2 &\equiv 0 \mod 2 \\
            0 + 0 + 1 &\equiv 0 \mod 2
        \end{align*}
        And this is a contradiction.
        \item If $q$ is even and $p$ is odd, then \begin{align*}
            ap^2 + bpq + cq^2 &\equiv 0 \mod 2 \\
            1 + 0 + 0 &\equiv 0 \mod 2
        \end{align*}
        And this is a contradiction.
        \item If $p,q$ are both odd, \begin{align*}
             ap^2 + bpq + cq^2 &\equiv 0 \mod 2 \\
             1 + 1 + 1 &\equiv 0 \mod 2 \\
             1 &\equiv 0 \mod 2
        \end{align*}
        And this is a contradiction.
    \end{itemize}
    All cases yield contradictions and hence the result follows.
    \end{proof}
    \item Let $\mathbb{Z}_p$ be the set $\set{[0],[1],[2],[3],\ldots,[p-1]}$ with the usual modular arithmetic. Prove that if $p$ is prime, then for all $[a] \neq [0] \in \mathbb{Z}_p$, there exists $[a]^{-1}$ such that $[a] \cdot [a]^{-1} = [1]$ (Note that $\cdot$ is multiplication in modular arithmetic; Hint: Bézout's identity) \begin{proof}
        Let $p$ be prime $[a] \neq [0] \in \mathbb{Z}_p$. Then, $[a] = [m]$ for some $1 < m < p$ and since $p$ is prime, $\gcd (m,p) = 1$. By Bézout's identity, we know there exists $x,y \in \mathbb{Z}$ such that $mx + py = 1$ so $mx = 1-py$ and $mx \equiv 1 \mod p$. It follows that this $x$ is an $x$ such that $mx \equiv 1 \mod p$ and by the rules of modular arithmetic, we know $[a][x] = [m][x] = [mx] =[1]$ so we have found an $[a]^{-1}$ as required.
    \end{proof}
    \item Let $\mathbb{Z}_n$ be the set $\set{[0],[1],[2],[3],\ldots,[n-1]}$ under addition mod $n$ and suppose $M \subseteq \mathbb{Z}_n$ is a non-empty subset such that for all $[a],[b] \in M$, $[a] + [n-b] \in M$ (Note that $+$ denotes addition mod $n$). \begin{enumerate}
        \item Prove that $[0] \in M$. \begin{proof}
            Let everything be as stated. Since $M$ is non-empty, we know $[a] \in M$, so $[a] + [n-a] = [a+n-a] = [n] = [0] \in M$ as required.
        \end{proof}
        \item Prove that for all $[a] \in M$, $[n-a] \in M$.
        \begin{proof}
            Let everything be as stated. From Q7a, we know $[0] \in M$. Let $[a] \in M$. Then, from the definition, $[0] + [n-a] = [n-a] \in M$ and hence the result follows.
        \end{proof}
        \item Prove that $|M|$ divides $n$. (Hint: consider equivalence classes under the relation where for all $[a], [b] \in \mathbb{Z}_n$, $[a] \sim [b]$ if and only if $[a] + [n-b] \in M$) \begin{proof}
            Let everything be as stated. For all $[c] \notin M$ but $[c] \in \mathbb{Z}_n$, construct a set $[c]+M$ where for all $[x] \in [c] + M$, $[x] = [c] + [a]$ for some $[a] \in M$. Notice by construction, $[c] + M$ has $|M|$ elements and every element in $\mathbb{Z}_n$ lies in some $[c] + M$. Notice if $[y] \in [c]+M$ and $[y] \in [d] + M$ for some $[c],[d] \in \mathbb{Z}_n$, then $[y] = [c] + [a_0] = [d] + [a_1]$ where $[a_0],[a_1] \in M$ so $[c] = [d] + [a_0] + [a_1] = d + [a_0+a_1]$ so $[c] \in [d] + M$ and the same argument could be applied backwards, yielding $[c] + M = [d] + M$. It follows that $[c]+M,[d]+M$ are disjoint or the same for all $[c],[d] \in \mathbb{Z}_n$. Hence, we know that $\mathbb{Z}_n$ is a disjoint union of sets of the form $[c] + M$, all of which have $|M|$ elements. It follows that the number of elements in $\mathbb{Z}_n$, which is $n$, is equal to $\ell|M|$ for some $\ell \in \mathbb{Z}$ so $|M|$ divides $n$ as required.
        \end{proof}
    \end{enumerate}
    \item We will prove Fermat's little theorem, i.e. for all $a ,p \in \mathbb{Z}$ such that $p$ is prime and $p \nmid a$, $a^{p-1} \equiv 1 \mod p$. Let $\mathbb{Z}_p$ be the set $\set{[0],[1],[2],[3],\ldots,[p-1]}$ with the usual modular arithmetic. By Q6, we know every element in the set $\mathbb{Z}_p \backslash \set{[0]}$ is invertible under multiplication. \begin{enumerate}
        \item Assume $k$ is the smallest natural number such that $a^k \equiv 1 \mod p$ for some $a \in \mathbb{Z}$ such that $1 \leq a \leq p-1$. Let $S \subseteq \mathbb{Z}_p \backslash \set{[0]}$ be a non-empty subset such that $S = \set{[1],[a],[a]^2, \ldots, [a]^{k-1}}$. Similarly to Q7c, prove that $|S| = k$ divides $|\mathbb{Z}_p \backslash \set{[0]}| = p-1$. \begin{proof}
            Let everything be as stated and for all $[c] \in \mathbb{Z}_p \backslash \set{[0]}$, construct a set $[c]S$ where its elements are defined as $[c] \cdot [a]$ for every $[a] \in S$. It follows that $[c] S$ has $|S| = k$ elements. Notice if $[x] \in [c]S$ and $[x] \in [d]S$, $[x] = [c][a_0] = [d][a_1]$ where $[a_0],[a_1] \in S$ so $[c] = [d][a_1][a_0] = [d][a_1 \cdot a_0]$ and thus $[c] \in [d]S$ and the same argument could be applied backwards, yielding $[c]M = [d]M$. It follows that that $[c]M,[d]M$ are disjoint or the same for all $[c],[d] \in \mathbb{Z}_p \backslash \set{[0]}$. It follows that the number of elements in $\mathbb{Z}_p \backslash \set{[0]}$, which is $p-1$, is equal to $m |S| = km$ for some $m \in \mathbb{Z}$ so $k \mid p-1$ as required.
        \end{proof}
        \item Hence, prove that $a^{p-1} \equiv 1 \mod p$. \begin{proof}
            We know from Q8a that $a^{p-1} = a^{km}$ where $k$ is the smallest natural number such that $a^k \equiv 1 \mod p$ and $m \in \mathbb{Z}$. It follows that $a^{p-1}\equiv a^{km} \equiv 1^m \equiv 1 \mod p$ so $a^{p-1} \equiv 1 \mod p$ as required.
        \end{proof}
    \end{enumerate}
\end{enumerate}
\end{document}
