

\documentclass[letterpaper,12pt]{article}
\newcommand{\hw}{2} 
\usepackage{amsmath, amsfonts, amssymb, amsthm}
\usepackage[paper=letterpaper,left=25mm,right=25mm,top=3cm,bottom=25mm]{geometry}
\usepackage{fancyhdr} %% for details on how this work, search-engine ``fancyhdr documentation''
\pagestyle{fancy}
\usepackage{array}
\usepackage{marginnote}
\lhead{MATH 220 Practice Midterm} % course name as top-left
\chead{Page \thepage \ of 8} % homework number in top-centre
\rhead{Student No: \ \ \ \ \ \ \ \ \ \ \ \ \ \ \ \ \ \ \ \ }

\cfoot{Page \thepage \ of 8} % page in middle
\usepackage{ragged2e}

\renewcommand{\headrulewidth}{0.4pt}
\renewcommand{\footrulewidth}{0.4pt}

\newcommand{\set}[1]{\left\{ #1 \right\}}
%% We also redfine the negation symbol:
\renewcommand{\neg}{\sim}
\newtheorem{lemma}{Lemma}[section]
\theoremstyle{definition}
\newtheorem{theorem}{Theorem}[section]
\makeatletter
\newsavebox\myboxA
\newsavebox\myboxB
\newcolumntype{P}[1]{>{\centering\arraybackslash}p{#1}}
\newlength\mylenA
\newcommand*\xoverline[2][0.75]{%
    \sbox{\myboxA}{$\m@th#2$}%
    \setbox\myboxB\null% Phantom box
    \ht\myboxB=\ht\myboxA%
    \dp\myboxB=\dp\myboxA%
    \wd\myboxB=#1\wd\myboxA% Scale phantom
    \sbox\myboxB{$\m@th\overline{\copy\myboxB}$}%  Overlined phantom
    \setlength\mylenA{\the\wd\myboxA}%   calc width diff
    \addtolength\mylenA{-\the\wd\myboxB}%
    \ifdim\wd\myboxB<\wd\myboxA%
       \rlap{\hskip 0.5\mylenA\usebox\myboxB}{\usebox\myboxA}%
    \else
        \hskip -0.5\mylenA\rlap{\usebox\myboxA}{\hskip 0.5\mylenA\usebox\myboxB}%
    \fi}
\makeatother
\begin{document}
\centering
 \textbf{MATH 220 Practice Midterm --- September, 2024, Duration: 50 minutes}
 \\
\textit{This test has \textbf{5 questions} on \textbf{8 pages}, for a total of 50 points. }
\vspace{2cm}
\renewcommand{\arraystretch}{2}
\\
\begin{tabular}{ | m{7.5cm}| m{7.5cm}| } 
  \hline
  First Name: & Last Name: \\
  \hline
  Student Number: & Section: \\
  \hline 
 \multicolumn{2}{| l |}{Signature:}  \\
  \hline
\end{tabular}
\\
\vspace{1.5cm}
\begin{tabular}{ | P{1.7cm} | P{0.4cm}| P{0.4cm}|P{0.4cm}|P{0.4cm}|P{0.4cm}|P{0.4cm}|P{0.4cm}|P{0.4cm}|P{0.4cm}|P{0.4cm}|P{0.4cm}|P{0.4cm}|P{0.4cm}|P{0.4cm}} 
  \hline
 Question: &1 & 2&3&4&5 \\
 \hline
 Points: & & & & &    \\
  \hline
  Total:  & \multicolumn{5}{| r |}{/50} \\
  \hline
\end{tabular}
\clearpage
\begin{enumerate}
    \item[1.] \reversemarginpar\marginnote{ \fbox{10 Marks} }[-0.24in] Negate each of the following and prove or disprove the original statement: \begin{enumerate}
        \item For all $x \in \mathbb{R}$, there exists $q \in \mathbb{Q}$ such that for all $r < q \in \mathbb{Q}$, $r+q < x$.
    \end{enumerate}
\end{enumerate}
\pagebreak
\begin{enumerate}
    \item[] \begin{enumerate}
        \item[(b)] For all $x \in \mathbb{R}$, there exists $y \in \mathbb{R}$ such that for all $z \in \mathbb{R}$, if $xy < z$ then $x < 0$ or $x^2 + y^2 < z$ 
    \end{enumerate}
\end{enumerate}
\pagebreak
\begin{enumerate}
    \item[2.]  \reversemarginpar\marginnote{ \fbox{10 Marks} }[-0.24in] Let $p \in \mathbb{N}$ and assume $p > 1$. Prove that if there exists $x \in \mathbb{Z}$ such that $x \not \equiv 0 \mod p$ and for all $y \in \mathbb{Z}$, $xy \not \equiv 1 \mod p$, then $p$ is not prime.
\end{enumerate}
\pagebreak
\begin{enumerate}
    \item[3. ]\reversemarginpar\marginnote{ \fbox{10 Marks} }[-0.24in] Let $a,b \in \mathbb{Z}$ where $a,b \neq 0$ and $S = \set{ax + by | x,y \in \mathbb{Z}, ax + by > 0}$. You may not use Bézout's lemma for this section. \begin{enumerate}
        \item Prove that $S$ is non-empty. 
            \vspace{1.1in}
        \item Prove that the minimal element $d = as + bt \in S$ for some $s,t \in \mathbb{Z}$ divides both $a$ and $b$. (Hint: Euclidean division of $a$ by $d$ and $b$ by $d$)
    \end{enumerate}
\end{enumerate}
\pagebreak
\begin{enumerate}
    \item[] \begin{enumerate}
        \item[(c)] Prove that if $c \mid a$ and $c \mid b$, then $c \leq d$.
    \end{enumerate}
\end{enumerate}
\pagebreak
\begin{enumerate}
    \item[4.]\reversemarginpar\marginnote{ \fbox{10 Marks} }[-0.24in] 
  Let $A\subseteq \mathbb{Q}$ and $S = \set{c_1a_1 + c_2a_2 + c_3a_3 \ldots | a_1,a_2,,a_3\ldots \in A, c_1,c_2,c_3 \ldots \in \mathbb{Z}}$. Prove by induction on the elements of $A$ that if $A$ is finite and non-empty, then there exists $q \in \mathbb{Q}$ such that $S = \set{cq, c \in \mathbb{Z}}$. (Hint: Show that $S \subseteq \set{cq, c \in \mathbb{Z}}$ and $S \supseteq \set{cq, c \in \mathbb{Z}}$ and find $q$)
\end{enumerate}
\pagebreak
\begin{enumerate}
    \item[5.] \reversemarginpar\marginnote{ \fbox{10 Marks} }[-0.24in] Prove or disprove that the sequence $(x_n)_{n \in \mathbb{N}} = \frac{1\cdot 3 \cdot 5 \cdot \ldots \cdot (2n-1)}{2 \cdot 4 \cdot 6 \cdot \ldots (2n)}$ converges. (Hint: Consider an inequality between this sequence and $\frac{1}{\sqrt{3n+1}}$)
\end{enumerate}
\end{document}
