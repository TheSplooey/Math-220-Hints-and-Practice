\documentclass[letterpaper,12pt]{article}
\newcommand{\hw}{2} 
\usepackage{amsmath, amsfonts, amssymb, amsthm}
\usepackage[paper=letterpaper,left=25mm,right=25mm,top=3cm,bottom=25mm]{geometry}
\usepackage{fancyhdr} %% for details on how this work, search-engine ``fancyhdr documentation''
\pagestyle{fancy}
\usepackage{array}
\usepackage{marginnote}
\lhead{MATH 220 Practice Midterm 2} % course name as top-left
\chead{Page \thepage \ of 10} % homework number in top-centre
\rhead{Student No: \ \ \ \ \ \ \ \ \ \ \ \ \ \ \ \ \ \ \ \ }

\cfoot{Page \thepage \ of 10} % page in middle
\usepackage{ragged2e}

\renewcommand{\headrulewidth}{0.4pt}
\renewcommand{\footrulewidth}{0.4pt}

\newcommand{\set}[1]{\left\{ #1 \right\}}
%% We also redfine the negation symbol:
\renewcommand{\neg}{\sim}
\newtheorem{lemma}{Lemma}[section]
\theoremstyle{definition}
\newtheorem{theorem}{Theorem}[section]
\makeatletter
\newsavebox\myboxA
\newsavebox\myboxB
\newcolumntype{P}[1]{>{\centering\arraybackslash}p{#1}}
\newlength\mylenA
\newcommand*\xoverline[2][0.75]{%
    \sbox{\myboxA}{$\m@th#2$}%
    \setbox\myboxB\null% Phantom box
    \ht\myboxB=\ht\myboxA%
    \dp\myboxB=\dp\myboxA%
    \wd\myboxB=#1\wd\myboxA% Scale phantom
    \sbox\myboxB{$\m@th\overline{\copy\myboxB}$}%  Overlined phantom
    \setlength\mylenA{\the\wd\myboxA}%   calc width diff
    \addtolength\mylenA{-\the\wd\myboxB}%
    \ifdim\wd\myboxB<\wd\myboxA%
       \rlap{\hskip 0.5\mylenA\usebox\myboxB}{\usebox\myboxA}%
    \else
        \hskip -0.5\mylenA\rlap{\usebox\myboxA}{\hskip 0.5\mylenA\usebox\myboxB}%
    \fi}
\makeatother
\begin{document}
\centering
 \textbf{MATH 220 Practice Midterm 2 --- September, 2024, Duration: 50 minutes}
 \\
\textit{This test has \textbf{5 questions} on \textbf{10 pages}, for a total of 50 points. }
\vspace{2cm}
\renewcommand{\arraystretch}{2}
\\
\begin{tabular}{ | m{7.5cm}| m{7.5cm}| } 
  \hline
  First Name: & Last Name: \\
  \hline
  Student Number: & Section: \\
  \hline 
 \multicolumn{2}{| l |}{Signature:}  \\
  \hline
\end{tabular}
\\
\vspace{1.5cm}
\begin{tabular}{ | P{1.7cm} | P{0.4cm}| P{0.4cm}|P{0.4cm}|P{0.4cm}|P{0.4cm}|P{0.4cm}|P{0.4cm}|P{0.4cm}|P{0.4cm}|P{0.4cm}|P{0.4cm}|P{0.4cm}|P{0.4cm}|P{0.4cm}} 
  \hline
 Question: &1 & 2&3&4&5 \\
 \hline
 Points: & & & & &    \\
  \hline
  Total:  & \multicolumn{5}{| r |}{/50} \\
  \hline
\end{tabular}
\clearpage
\begin{enumerate}
    \item[1.] \reversemarginpar\marginnote{ \fbox{10 Marks} }[-0.24in] Negate each of the following and prove or disprove the original statement: \begin{enumerate}
        \item For all $x \in \mathbb{Z}$, there exists $y\in \mathbb{Z}$, such that there exists $z \in \mathbb{Z}$, such that $xy > z$ implies $ x+y < z$.
    \end{enumerate}
\end{enumerate}
\pagebreak
\begin{enumerate}
    \item[] \begin{enumerate}
        \item[(b)] For all $x \in \mathbb{N}$, for all $y \in \mathbb{N}$ such that $x< y$, there exists $a,b \in \mathbb{Z}$ such that $ax + by < \left | \frac{x}{y} \right |$
    \end{enumerate}
\end{enumerate}
\pagebreak
\begin{enumerate}
    \item[2. ]\reversemarginpar\marginnote{ \fbox{10 Marks} }[-0.24in] Let $n,k \in \mathbb{N}$. Prove that if for all $m \in \mathbb{Z}$, $m^k \neq n$, then $n^{1/k}$ is irrational.
\end{enumerate}
\pagebreak
\begin{enumerate}
    \item[3.]  \reversemarginpar\marginnote{ \fbox{10 Marks} }[-0.24in] The Archimedean property of the reals guarantees that for all $x,y \in \mathbb{R}$ where $x > 0$, there exists $n \in \mathbb{N}$ such that $nx > y$. \begin{enumerate}
        \item Let $x,y \in \mathbb{R}^+$. Prove that there exists $m \in \mathbb{N}$ such that $(m-1)x \leq y < mx$.
        \vspace{2.4in}
        \item Using (a), prove that for all $x,y \in \mathbb{R}^+$ such that $x<y$, there exists $q \in \mathbb{Q}$ such that $x < q < y$. (Hint: Consider $y-x$ and also notice $1 \in \mathbb{R}^+$)
    \end{enumerate}
\end{enumerate}
\pagebreak
\begin{enumerate}
    \item[] \begin{enumerate}
        \item[(c)] Assume that $\sqrt{2}$ is irrational. Using $\sqrt{2}$, prove that for all $x,y \in \mathbb{R}^+$ such that $x<y$, there exists $z \in \mathbb{R}\backslash \mathbb{Q}$ such that $x<z<y$.You may assume that irrational numbers added to and multiplied by rational numbers are still irrational.  (Hint: From part (b), we have that there exists $p,q \in \mathbb{Q}$ such that $x<p<q<y$)
    \end{enumerate}
\end{enumerate}
\pagebreak
\begin{enumerate}
    \item[4.]
 \begin{enumerate}
     \item  \reversemarginpar\marginnote{ \fbox{10 Marks} }[-0.24in] 
 Let $A = \set{-1, 2, \frac{1}{2}}$. Prove by induction that if 1 is written as a product $1=p_1p_2p_3\ldots p_n$ where $p_i \in A$, then $n$ is even. 
 \end{enumerate}
\end{enumerate}
\pagebreak
\begin{enumerate}
    \item[] \begin{enumerate}
        \item[(b)] Prove or disprove that the same applies for $B = \set{-1, \pm 2, \pm \frac{1}{2}}$.
    \end{enumerate}
\end{enumerate}
\pagebreak
\begin{enumerate}
    \item[5.]  \begin{enumerate}
        \item \reversemarginpar\marginnote{ \fbox{10 Marks} }[-0.24in] Prove that for all $\delta > 0$, there exists $n \in \mathbb{N}$ such that $\frac{1}{\frac{\pi}{2}+2\pi n} < \delta$ and likewise, there exists $m \in \mathbb{N}$ such that  $\frac{1}{\frac{-\pi}{2}+2\pi m} < \delta$ (Hint: Archimedean property of the reals)
    \end{enumerate}
\end{enumerate}
\pagebreak
\begin{enumerate}
    \item[] \begin{enumerate}
        \item[(b)] Hence prove that the limit $\lim_{x \rightarrow 0} \sin \left (\frac{1}{x} \right ) = L$ does not exist. 
    \end{enumerate}
\end{enumerate}
\end{document}
