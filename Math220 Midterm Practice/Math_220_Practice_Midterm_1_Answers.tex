

\documentclass[letterpaper,12pt]{article}
\newcommand{\hw}{2} 
\usepackage{amsmath, amsfonts, amssymb, amsthm}
\usepackage[paper=letterpaper,left=25mm,right=25mm,top=3cm,bottom=25mm]{geometry}
\usepackage{fancyhdr} %% for details on how this work, search-engine ``fancyhdr documentation''
\pagestyle{fancy}
\usepackage{array}
\usepackage{marginnote}
\lhead{MATH 220 Practice Midterm} % course name as top-left
\chead{Page \thepage \ of X} % homework number in top-centre
\rhead{Answer Key}

\cfoot{Page \thepage \ of X} % page in middle
\usepackage{ragged2e}

\renewcommand{\headrulewidth}{0.4pt}
\renewcommand{\footrulewidth}{0.4pt}

\newcommand{\set}[1]{\left\{ #1 \right\}}
%% We also redfine the negation symbol:
\renewcommand{\neg}{\sim}
\newtheorem{lemma}{Lemma}[section]
\theoremstyle{definition}
\newtheorem{theorem}{Theorem}[section]
\makeatletter
\newsavebox\myboxA
\newsavebox\myboxB
\newcolumntype{P}[1]{>{\centering\arraybackslash}p{#1}}
\newlength\mylenA
\newcommand*\xoverline[2][0.75]{%
    \sbox{\myboxA}{$\m@th#2$}%
    \setbox\myboxB\null% Phantom box
    \ht\myboxB=\ht\myboxA%
    \dp\myboxB=\dp\myboxA%
    \wd\myboxB=#1\wd\myboxA% Scale phantom
    \sbox\myboxB{$\m@th\overline{\copy\myboxB}$}%  Overlined phantom
    \setlength\mylenA{\the\wd\myboxA}%   calc width diff
    \addtolength\mylenA{-\the\wd\myboxB}%
    \ifdim\wd\myboxB<\wd\myboxA%
       \rlap{\hskip 0.5\mylenA\usebox\myboxB}{\usebox\myboxA}%
    \else
        \hskip -0.5\mylenA\rlap{\usebox\myboxA}{\hskip 0.5\mylenA\usebox\myboxB}%
    \fi}
\makeatother
\usepackage[]{mdframed}


% draw a frame around given text
\newcommand{\framedtext}[1]{%
\par%
\noindent\fbox{%
    \parbox{\dimexpr\linewidth-2\fboxsep-2\fboxrule}{#1}%
}%
}
\begin{document}
\centering
 \textbf{MATH 220 Practice Midterm --- September, 2024, Duration: 50 minutes}
 \\
\textit{This test has \textbf{5 questions} on \textbf{8 pages}, for a total of 50 points. }
\vspace{2cm}
\renewcommand{\arraystretch}{2}
\\
\begin{tabular}{ | m{7.5cm}| m{7.5cm}| } 
  \hline
  First Name: & Last Name: \\
  \hline
  Student Number: & Section: \\
  \hline 
 \multicolumn{2}{| l |}{Signature:}  \\
  \hline
\end{tabular}
\\
\vspace{1.5cm}
\begin{tabular}{ | P{1.7cm} | P{0.4cm}| P{0.4cm}|P{0.4cm}|P{0.4cm}|P{0.4cm}|P{0.4cm}|P{0.4cm}|P{0.4cm}|P{0.4cm}|P{0.4cm}|P{0.4cm}|P{0.4cm}|P{0.4cm}|P{0.4cm}} 
  \hline
 Question: &1 & 2&3&4&5 \\
 \hline
 Points: & & & & &    \\
  \hline
  Total:  & \multicolumn{5}{| r |}{/50} \\
  \hline
\end{tabular}
\clearpage
\begin{enumerate}
    \item[1.] \reversemarginpar\marginnote{ \fbox{10 Marks} }[-0.24in] Negate each of the following and prove or disprove the original statement: \begin{enumerate}
        \item For all $x \in \mathbb{R}$, there exists $q \in \mathbb{Q}$ such that for all $r < q \in \mathbb{Q}$, $r+q < x$".  
    \end{enumerate}
    \begin{mdframed}
            \textbf{Solution:}
            The negation is "There exists $x \in \mathbb{R}$ such that for all $q \in \mathbb{Q}$, there exists $r < q \in \mathbb{Q}$ such that $r + q \geq x$.
            \begin{proof}
                Let $x \in \mathbb{R}$. Pick $q = \frac{\lfloor x \rfloor}{2} - 1 \in \mathbb{Q}$. Notice for all $r < q \in \mathbb{Q}$, $r+q < 2q = \lfloor x \rfloor -2 \leq x - 2 < x$ so the result holds.
            \end{proof}
        \end{mdframed}
\end{enumerate}
\pagebreak
\begin{enumerate}
    \item[] \begin{enumerate}
        \item[(b)] For all $x \in \mathbb{R}$, there exists $y \in \mathbb{R}$ such that for all $z \in \mathbb{R}$, if $xy < z$ then $x < 0$ or $x^2 + y^2 < z$ 
    \end{enumerate}
    \begin{mdframed}
            \textbf{Solution:}
            The negation is "There exists $x \in \mathbb{R}$ such that for all $y \in \mathbb{R}$, there exists $z \in \mathbb{R}$ such that $xy < z$ and $x \geq 0$ and $x^2 + y^2 \geq z$".
            \renewcommand{\proofname}{Disproof} \begin{proof}
                Choose $x = 2$ and let $y \in \mathbb{R}$. Choose $z = 4 + y^2$. Notice $x \geq 0$ and $z \leq 4 + y^2 = x^2 + y^2$. Now, notice $y^2 - 2y + 4 = (y-1)^2 + 2 \geq 2 > 0$, so \begin{align*}
                    y^2 -2y + 4 &> 0 \\
                    y^2 + 4 &> 2y \\
                    z = x^2 + y^2 &> xy
                \end{align*}
                and thus $xy < z$. This proves the negation and thus disproves the original statement.
            \end{proof}
        \end{mdframed}
\end{enumerate}
\pagebreak
\begin{enumerate}
    \item[2.]  \reversemarginpar\marginnote{ \fbox{10 Marks} }[-0.24in] Let $p \in \mathbb{N}$ and assume $p > 1$. Prove that if there exists $x \in \mathbb{Z}$ such that $x \not \equiv 0 \mod p$ and for all $y \in \mathbb{Z}$, $xy \not \equiv 1 \mod p$, then $p$ is not prime.
    \begin{mdframed}
        \textbf{Solution:} \begin{proof}
            We prove the contrapositive. Let $p \in \mathbb{N}$ be prime. Let $x \in \mathbb{Z}$. \begin{itemize}
                \item If $x \equiv 0 \mod p$, then we are done.
                \item If $x \not \equiv 0 \mod p$, we know $x \equiv r \mod p$ for some $1 \leq r < p$. By Bézout's lemma, we know there exists $\alpha,\beta \in \mathbb{Z}$ such that $\alpha r + \beta p = \gcd (r,p) = 1$ so $\alpha r = 1 - \beta p$. It follows that $\alpha r \equiv 1 \mod p$ so choosing such $\alpha = y$ would yield $xy \equiv 1 \mod p$ as required.
            \end{itemize}
        \end{proof}
    \end{mdframed}
\end{enumerate}
\pagebreak
\begin{enumerate}
    \item[3. ]\reversemarginpar\marginnote{ \fbox{10 Marks} }[-0.24in] Let $a,b \in \mathbb{Z}$ where $a,b \neq 0$ and $S = \set{ax + by | x,y \in \mathbb{Z}, ax + by > 0}$. You may not use Bézout's lemma for this section. \begin{enumerate}
        \item Prove that $S$ is non-empty. 
            \begin{mdframed}
                \textbf{Solution:}
                \begin{proof}
                    Notice $a \in S$ so $S$ is non-empty.
                \end{proof}
            \end{mdframed}
            \vspace{0.2in}
        \item Prove that the minimal element $d = as + bt \in S$ for some $s,t \in \mathbb{Z}$ divides both $a$ and $b$. (Hint: Euclidean division of $a$ by $d$ and $b$ by $d$)
        \begin{mdframed}
            \textbf{Solution:} \begin{proof}
                Let $S$ be as stated and let $d$ be the minimal element of $S$. By Euclidean division, we know there exists $m, r \in \mathbb{Z}$ such that $a = md + r$, $0 \leq r < d$. Notice $r = a - md = a - mas - mbt = a(1-ms) + b(-mt)$ so $r \in S \cup \set{0}$. By assumption, we have $d$ is the minimal element in $S$, but $0 \leq r < d$ so $r = 0$. Likewise, there exists $n, r \in \mathbb{Z}$ such that $b = nd + r$, $0 \leq r < d$. It follows that $r = a(-ns) + b(1-nt)$ so $r \in S \cup \set{0}$, and by $d$ is the minimal element, we have $r = 0$. It follows that $a = md$ and $b = nd$ so $d \mid a$ and $d \mid b$.
            \end{proof}
        \end{mdframed}
    \end{enumerate}
\end{enumerate}
\pagebreak
\begin{enumerate}
    \item[] \begin{enumerate}
        \item[(c)] Prove that if $c \mid a$ and $c \mid b$, then $c \leq d$.
    \end{enumerate}
    \begin{mdframed}
        \textbf{Solution:} \begin{proof}
            Let $S$ and $d$ be as stated and assume $c \mid a$ and $c \mid b$, so $a = ck$ and $b = c\ell$ for some $k,\ell \in \mathbb{Z}$. Notice $d = as + bt = c(sk + t\ell)$ so $c \mid d$. By assumption, since $d \in S$, $d > 0$, so $c \leq d$ as required.
        \end{proof}
    \end{mdframed}
\end{enumerate}
\pagebreak
\begin{enumerate}
    \item[4.]\reversemarginpar\marginnote{ \fbox{10 Marks} }[-0.24in] 
  Let $A\subseteq \mathbb{Q}$ and $S = \set{c_1a_1 + c_2a_2 + c_3a_3 \ldots | a_1,a_2,,a_3\ldots \in A, c_1,c_2,c_3 \ldots \in \mathbb{Z}}$. Prove by induction on the elements of $A$ that if $A$ is finite and non-empty, then there exists $q \in \mathbb{Q}$ such that $S = \set{cq, c \in \mathbb{Z}}$. (Hint: Show that $S \subseteq \set{cq, c \in \mathbb{Z}}$ and $S \supseteq \set{cq, c \in \mathbb{Z}}$ and find $q$)
  \begin{mdframed}
  \textbf{Solution:}
      \begin{proof}
          Let $A$ and $S$ be as stated. We proceed with mathematical induction on the number of elements in $A$. \begin{itemize}
        \item If $A$ has 1 element $a$, then, $S = \set{ca, c \in \mathbb{Z}}$ and since $a \in \mathbb{Q}$, the base case holds.
        \item Assume there exists $q \in \mathbb{Q}$ such that $S = \set{cq, c \in \mathbb{Z}}$ where $A=A_{k}$ has $k$ elements. Consider $A=A_{k+1}$ and partition $A$ into $A\backslash \set{a}$ and $\set{a}$ for some $a \in A_{k+1}$. Notice $S$ consists of all the ways multiples of $a$ can add to multiples of elements in $ A\backslash \set{a}$, which is precisely elements in $\set{c_1 a + c_2 q, c_1,c_2 \in \mathbb{Z}}$. Since $a,q \in \mathbb{Q}$, $a = \frac{x_1}{y_1}$ and $q = \frac{x_2}{y_2}$ where $y_1,y_2 \neq 0$. Let $\frac{\gcd (y_2x_1,y_1x_2)}{y_1y_2} = u$. We claim that $\set{ cu, c \in \mathbb{Z} } = S$. \begin{itemize}
            \item Let $p \in \set{ cu, c \in \mathbb{Z} } = S$. By Bézout's lemma, we know there exists integers $\alpha,\beta$ such that \begin{align*}
                p = cu &= \frac{c( \alpha y_2x_1 + \beta y_1x_2)}{y_1y_2} \\
                &= \frac{c\alpha y_2x_1 + c\beta y_1x_2}{y_1y_2} \\
                &= \frac{c\alpha x_1}{y_1} + \frac{c\beta x_2}{y_2} \\
                &= c\alpha(a) + c\beta(q)
            \end{align*}
            so $p \in \set{c_1 a + c_2 q, c_1,c_2 \in \mathbb{Z}} = S$, $\set{ cu, c \in \mathbb{Z} } \subseteq S$.
            \item Let $p \in S$ so $p = c_1a + c_2 q$ for some $c_1, c_2 \in \mathbb{Z}$. Then, \begin{align*}
                p &= \frac{c_1x_1}{y_1} + \frac{c_2 x_2}{y_2} \\
                &= \frac{c_1y_2x_1 + c_2 y_1x_2}{y_1y_2}
            \end{align*}
            Notice $\gcd (y_2x_1,y_1x_2) \mid y_2x_1$ and $\gcd (y_2x_1,y_1x_2) \mid y_1x_2$, so $$y_2x_1 = \alpha \gcd (y_2x_1,y_1x_2)$$ $$y_1x_2 = \beta \gcd (y_2x_1,y_1x_2)$$ where $\alpha,\beta \in \mathbb{Z}$ and thus
            \begin{align*}
                p &= \frac{c_1\alpha \gcd (y_2x_1,y_1x_2)  + c_2 \beta \gcd (y_2x_1,y_1x_2)}{y_1y_2} \\
                &= (c_1\alpha + c_2 \beta) u
            \end{align*}
            so $p \in \set{ cu, c \in \mathbb{Z} }$, $\set{ cu, c \in \mathbb{Z} } \supseteq S$.
        \end{itemize}
        So there exists $q \in \mathbb{Q}$ such that $\S = \set{cq, c \in \mathbb{Z}}$ as required.
    \end{itemize}
    Since the base case and the inductive step hold, it follows that for any finite and non-empty $A$, there exists $q \in \mathbb{Q}$ such that $S = \set{cq, c \in \mathbb{Z}}$.
      \end{proof}
  \end{mdframed}
\end{enumerate}
\pagebreak
\begin{enumerate}
    \item[5.] \reversemarginpar\marginnote{ \fbox{10 Marks} }[-0.24in] Prove or disprove that the sequence $(x_n)_{n \in \mathbb{N}} = \frac{1\cdot 3 \cdot 5 \cdot \ldots \cdot (2n-1)}{2 \cdot 4 \cdot 6 \cdot \ldots (2n)}$ converges. (Hint: Consider an inequality between this sequence and $\frac{1}{\sqrt{3n+1}}$)
\begin{mdframed}
\textbf{Solution:}
    \begin{proof}
        We first prove that for all $n \in \mathbb{N}$, $\frac{1\cdot 3 \cdot 5 \cdot \ldots \cdot (2n-1)}{2 \cdot 4 \cdot 6 \cdot \ldots (2n)} \leq \frac{1}{\sqrt{3n+1}}$. We proceed with induction. \begin{itemize}
            \item For the base case, $\frac{1}{2} = \frac{1}{\sqrt{4}} =\frac{1}{\sqrt{3+ 1}}$ so the base case holds.
            \item Assume that the result holds for $n = k$. Then, for $k+1$, by assumption,
            \begin{align*}
              \frac{1\cdot 3 \cdot 5 \cdot \ldots \cdot (2n-1)(2n+1)}{2 \cdot 4 \cdot 6 \cdot \ldots (2n)(2n+2)}  &\leq \frac{(2n+1)}{\sqrt{3n-1}(2n+2)} \\
              &= \sqrt{\frac{(2n+1)^2}{(3n+1)(2n+2)^2}} \\
              &=  \sqrt{\frac{(2n+2)^2}{12n^3+28n^2+20n+4}} \\
              &\leq \sqrt{\frac{(2n+2)^2}{12n^3+28n^2+19n+4}} \\
              &= \sqrt{\frac{1}{(3n+4)}}
            \end{align*}
            so the inductive step holds
        \end{itemize}
        Since the base case and inductive step hold, this proves that $n \in \mathbb{N}$, $\frac{1\cdot 3 \cdot 5 \cdot \ldots \cdot (2n-1)}{2 \cdot 4 \cdot 6 \cdot \ldots (2n)} \leq \frac{1}{\sqrt{3n+1}}$. Now, let $\varepsilon > 0$ and choose $N = \left \lceil\frac{ 1  }{3\varepsilon^2}\right \rceil \in \mathbb{N}$. Notice, for all $n > N$, \begin{align*}
             \left | \frac{1\cdot 3 \cdot 5 \cdot \ldots \cdot (2n-1)}{2 \cdot 4 \cdot 6 \cdot \ldots (2n)}\right | &\leq   \left |\frac{1\cdot 3 \cdot 5 \cdot \ldots \cdot (2N-1)}{2 \cdot 4 \cdot 6 \cdot \ldots (2N)} \right| \\
             &\leq \left | \frac{1}{\sqrt{3N+1}} \right | \\
             &<  \frac{1}{\sqrt{3N}} \\
              &= \frac{1}{\sqrt{3\left \lceil\frac{ 1 }{3\varepsilon^2}\right \rceil }} \\
              &\leq \frac{1}{\sqrt{ \frac{ 3  }{3\varepsilon^2} }} = \varepsilon
        \end{align*}
        So $(x_n)_{n\in \mathbb{N}}$ converges as required.
    \end{proof}
    \end{mdframed}
\end{enumerate}
\end{document}
