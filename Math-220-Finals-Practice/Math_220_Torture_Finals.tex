\documentclass[letterpaper,12pt]{article}
\newcommand{\hw}{2} 
\usepackage{amsmath, amsfonts, amssymb, amsthm}
\usepackage[paper=letterpaper,left=25mm,right=25mm,top=3cm,bottom=25mm]{geometry}
\usepackage{fancyhdr} %% for details on how this work, search-engine ``fancyhdr documentation''
\pagestyle{fancy}
\usepackage{array}
\usepackage{marginnote}
\lhead{MATH 220 Torture Finals} % course name as top-left
\chead{Page \thepage \ of 25} % homework number in top-centre
\rhead{Student No: \ \ \ \ \ \ \ \ \ \ \ \ \ \ \ \ \ \ \ \ }

\cfoot{Page \thepage \ of 25} % page in middle
\usepackage{ragged2e}

\renewcommand{\headrulewidth}{0.4pt}
\renewcommand{\footrulewidth}{0.4pt}

\newcommand{\set}[1]{\left\{ #1 \right\}}
%% We also redfine the negation symbol:
\renewcommand{\neg}{\sim}
\newtheorem{lemma}{Lemma}[section]
\theoremstyle{definition}
\newtheorem{theorem}{Theorem}[section]
\makeatletter
\newsavebox\myboxA
\newsavebox\myboxB
\newcolumntype{P}[1]{>{\centering\arraybackslash}p{#1}}
\newlength\mylenA
\newcommand*\xoverline[2][0.75]{%
    \sbox{\myboxA}{$\m@th#2$}%
    \setbox\myboxB\null% Phantom box
    \ht\myboxB=\ht\myboxA%
    \dp\myboxB=\dp\myboxA%
    \wd\myboxB=#1\wd\myboxA% Scale phantom
    \sbox\myboxB{$\m@th\overline{\copy\myboxB}$}%  Overlined phantom
    \setlength\mylenA{\the\wd\myboxA}%   calc width diff
    \addtolength\mylenA{-\the\wd\myboxB}%
    \ifdim\wd\myboxB<\wd\myboxA%
       \rlap{\hskip 0.5\mylenA\usebox\myboxB}{\usebox\myboxA}%
    \else
        \hskip -0.5\mylenA\rlap{\usebox\myboxA}{\hskip 0.5\mylenA\usebox\myboxB}%
    \fi}
\makeatother
\begin{document}
\centering
 \textbf{MATH 220 Torture Finals --- October, 2024, Duration: 2.5 hours}
 \\
\textit{This test has \textbf{8 questions} on \textbf{25 pages}, for a total of 100 points. }
\vspace{2cm}
\renewcommand{\arraystretch}{2}
\\
\begin{tabular}{ | m{7.5cm}| m{7.5cm}| } 
  \hline
  First Name: & Last Name: \\
  \hline
  Student Number: & Section: \\
  \hline 
 \multicolumn{2}{| l |}{Signature:}  \\
  \hline
\end{tabular}
\\
\vspace{1.5cm}
\begin{tabular}{ | P{1.7cm} | P{0.4cm}| P{0.4cm}|P{0.4cm}|P{0.4cm}|P{0.4cm}|P{0.4cm}|P{0.4cm}|P{0.4cm}|P{0.4cm}|P{0.4cm}|P{0.4cm}|P{0.4cm}|P{0.4cm}|P{0.4cm}} 
  \hline
 Question: &1 & 2&3&4&5&6&7&8 \\
 \hline
 Points: & & & & & & & &      \\
  \hline
  Total:  & \multicolumn{8}{| r |}{/100} \\
  \hline
\end{tabular}
\clearpage
\begin{enumerate}
    \item[1.] \reversemarginpar\marginnote{ \fbox{10 Marks} }[-0.24in] Carefully define or restate each of the following: \begin{enumerate}
        \item A relation on $A$
        \vspace{1.4in}
        \item A function $f:A \mapsto B$
                \vspace{1.4in}
        \item The limit of a function $f: \mathbb{R} \mapsto \mathbb{R}$ as $x \rightarrow a$
                \vspace{1.4in}
        \item Euclidean division
                \vspace{1.4in}
        \item The principle of mathematical induction
    \end{enumerate}
\end{enumerate}
\pagebreak
\begin{enumerate}
    \item[2.] \reversemarginpar\marginnote{ \fbox{10 Marks} }[-0.24in] Write the negation of each of the following and prove or disprove the original statement.
    \begin{enumerate}
        \item There exists a prime $p \in \mathbb{Z}$ such that there exists $r,k \in \mathbb{Z}$ such that $0 \leq r \leq p-1$ and $k < p$ and $p \mid kr$.
    \end{enumerate}
\end{enumerate}
\pagebreak
\begin{enumerate}
    \item[] \begin{enumerate}
        \item[(b)] For all $n \in \mathbb{N}$, for all $p,q \in \mathbb{N}$ such that $k,\ell \in \mathbb{N}$ are the smallest integers such that $n \mid kp$ and $n \mid \ell q$, then if $k = \ell$, $|\set{ px \in \mathbb{Z} | 0 \leq px \leq kp \in \mathbb{Z}}| = |\set{ qy \in \mathbb{Z} | 0 \leq y q \leq \ell q \in \mathbb{Z}}|$
    \end{enumerate}
\end{enumerate}
\pagebreak
\begin{enumerate}
    \item[3. ]\reversemarginpar\marginnote{ \fbox{10 Marks} }[-0.24in] Prove that there does not exist a bijective $f: \mathbb{Q} \mapsto \mathbb{Q}\backslash \set{0}$ such that for all $x,y \in \mathbb{Q}$, $f(x+y) = f(x)f(y)$.
\end{enumerate}
\pagebreak
\begin{enumerate}
    \item[4.]\reversemarginpar\marginnote{ \fbox{10 Marks} }[-0.24in] Let $S = \set{1,2,3,\ldots,m \in \mathbb{N}}$ and $p_i:S \mapsto S$ be a function that swaps some $a,b \in S$ around where $a \neq b$ and fixes everything else, denoted $p_i = (ab)$. Prove that the identity map $e$ can only be written as a product $p_1p_2p_3\ldots p_n$ where $n$ is even by induction on the number of $p_i$.
\end{enumerate}
\pagebreak
\begin{enumerate}
    \item[5.]  \begin{enumerate}
    \item \reversemarginpar\marginnote{ \fbox{15 Marks} }[-0.24in] 
Let $D_n$ denote the set of rotations by an angle of $\frac{2\pi}{i}$ where $1 \leq n$ and reflections across an axis of symmetry on an $n-$gon, and let $\sim$ be a relation $\sigma \sim \tau$ if and only if $\sigma \tau$ is a rotation where $\sigma, \tau \in D_n$. Prove that $\sim$ is an equivalence relation.
\end{enumerate}
\end{enumerate}
\pagebreak
\begin{enumerate}
    \item[] \begin{enumerate}
        \item[(b)] Let $R \subseteq D_n$ be the set of rotations by an angle of $\frac{2\pi}{i}$ where $1 \leq n$, and for all $\sigma \in D_n$, let $\sigma R$ denote the set $\set{\sigma x | x \in R}$. Prove that the collection of $\sigma R$ forms a partition on $D_n$. 
    \end{enumerate}
\end{enumerate}
\pagebreak
\begin{enumerate}
    \item[]\begin{enumerate}
        \item[(c)] Prove that for all $\sigma \in D_n$, $\sigma R = [\sigma]$ where $[\sigma]$ is the equivalence class of $\sigma$ under $\sim$.
    \end{enumerate}
\end{enumerate}
\pagebreak
\begin{enumerate}
    \item[6.] \reversemarginpar\marginnote{ \fbox{10 Marks} }[-0.24in] Let $m,n$ be coprime integers with $m < n$. Prove that for all $a, b \in \mathbb{Z}$, there exists $x \in \mathbb{Z}$ such that $x \equiv a \mod m$ and $x \equiv b \mod m$. (Hint: Contradiction and pigeonhole principle)
\end{enumerate}
\pagebreak
\begin{enumerate}
    \item[7.] \reversemarginpar\marginnote{ \fbox{15 Marks} }[-0.24in] Prove or disprove each of the following: \begin{enumerate}
        \item There exists a bijective function $f:\mathbb{R} \mapsto S_1$ where $S_1$ is the unit circle in $\mathbb{C}$.
    \end{enumerate}
\end{enumerate}
\pagebreak
\begin{enumerate}
    \item[] \begin{enumerate}
        \item[(b)] There exists a function $f:\mathbb{R} \mapsto S_1$ where $S_1$ is the unit circle in $\mathbb{C}$ such that for all $x,y \in \mathbb{R}$, $f(x+y) = f(x)+f(y)$.
    \end{enumerate}
\end{enumerate}
\pagebreak
\begin{enumerate}
    \item[] \begin{enumerate}
        \item[(c)] A countable union of countable sets is countable.
    \end{enumerate}
\end{enumerate}
\pagebreak
\begin{enumerate}
    \item[] \begin{enumerate}
        \item[(d)] A denumerable intersection of uncountable sets is countable.
    \end{enumerate}
\end{enumerate}
\pagebreak
\begin{enumerate}
    \item[8.] \begin{enumerate}
    \item\reversemarginpar\marginnote{ \fbox{20 Marks} }[-0.24in]   Let $A \subseteq \mathbb{R}$. Prove that if $-A = \set{-a | a \in A}$, then $-\sup A = \inf A$. 
\end{enumerate}
\end{enumerate}
\pagebreak
\begin{enumerate}
   \item[] \begin{enumerate}
       \item[(b)] Let $\set{x_n} : \mathbb{N} \mapsto \mathbb{R}$ be a non-decreasing sequence. Prove that if $\set{x_n}$ is bounded above, then $\lim_{n \rightarrow \infty} x_n = \sup \set{x_n}$. 
   \end{enumerate}
\end{enumerate}
\pagebreak
\begin{enumerate}
\item[]
\begin{enumerate}
    \item[(c)] Prove that if $\set{x_n}$ is a non-increasing sequence, and if $\set{x_n}$ is bounded below, then $\lim_{n \rightarrow \infty} x_n = \inf \set{x_n}$
   \end{enumerate}
\end{enumerate}
\pagebreak
\begin{enumerate}
    \item[] \begin{enumerate}
        \item[(d)] Prove that if $\set{x_n} : \mathbb{N} \mapsto \mathbb{R}$ is bounded, then $\set{x_n}$ has a finite limit.
    \end{enumerate}
\end{enumerate}
\pagebreak
\begin{enumerate}
    \item[] \begin{enumerate}
        \item[(e)] Let $\set{x_n} : \mathbb{N} \mapsto \mathbb{R}$ be a sequence, and we define $\set{a_n}$ to be a subsequence of $\set{x_n}$ if we can obtain $\set{a_n}$ from removing some elements from $\set{x_n}$. Let $n \in \mathbb{N}$ be a peak if for all $m \geq n$, $x_m \geq x_n$. Prove that if $\set{x_n}$ has infinite peaks then it has a monotone (non-decreasing or non-increasing) subsequence. 
    \end{enumerate}
\end{enumerate}
\pagebreak
\begin{enumerate}
    \item[] \begin{enumerate}
        \item[(f)] Prove that if $\set{x_n}$ has finite peaks then it has a monotone subsequence. 
    \end{enumerate}
\end{enumerate}
\pagebreak
\begin{enumerate}
    \item[] \begin{enumerate}
        \item[(g)] Prove that if $\set{x_n}$ is bounded then there exists a convergent subsequence.
    \end{enumerate}
\end{enumerate}
\pagebreak
\begin{enumerate}
    \item[] \begin{enumerate}
        \item[(h)] We denote $\set{b_n} : \mathbb{N} \mapsto \mathbb{R}^n$ to be a sequence of vectors in $ \mathbb{R}^n$ and we say a sequence $\set{b_n} : \mathbb{N} \mapsto \mathbb{R}^n$ is bounded if there exists $M >0 \in \mathbb{R}$ such that for all $i \in \mathbb{N}$, $\|b_i \| \leq M$ where $\| b_i \|$ denotes the Euclidean norm of $b_i$. Prove that if $\set{b_n}$ which can be represented as $(\set{b_{n1}},\set{b_{n2}},\ldots,\set{b_{nn}})$ is bounded, then for all $j \in \mathbb{N}$, $\set{b_{nj}}$ is bounded.
    \end{enumerate}
\end{enumerate}
\pagebreak
\begin{enumerate}
    \item[] \begin{enumerate}
        \item[(i)] We say $\set{b_n} : \mathbb{N} \mapsto \mathbb{R}^n$ converges to $L \in \mathbb{R}^n$ if for all $\varepsilon > 0$, there exists $N \in \mathbb{N}$ such that for all $n > N$, $\|x_n - L \| < \varepsilon$ where $\|x_n-L\|$ denote the Euclidean norm. Prove that if $\set{b_n}$ has a convergent subsequence, then there exists a countable $K \subseteq I$ where $I$ is the indexed set of $\set{b_n}$
    \end{enumerate}
\end{enumerate}
\pagebreak
\begin{enumerate}
    \item[] \begin{enumerate}
        \item[(j)] Using the results from parts (g),(h),(i), prove that every bounded sequence $\set{b_n}: \mathbb{N} \mapsto \mathbb{R}^n$ has a convergent subsequence.
    \end{enumerate}
\end{enumerate}
\end{document}
