\documentclass[letterpaper,12pt]{article}
\newcommand{\hw}{2} 
\usepackage{amsmath, amsfonts, amssymb, amsthm}
\usepackage[paper=letterpaper,left=25mm,right=25mm,top=3cm,bottom=25mm]{geometry}
\usepackage{fancyhdr} %% for details on how this work, search-engine ``fancyhdr documentation''
\pagestyle{fancy}
\usepackage{array}
\usepackage{marginnote}
\lhead{MATH 220 Practice Finals 1} % course name as top-left
\chead{Page \thepage \ of 20} % homework number in top-centre
\rhead{Student No: \ \ \ \ \ \ \ \ \ \ \ \ \ \ \ \ \ \ \ \ }

\cfoot{Page \thepage \ of 20} % page in middle
\usepackage{ragged2e}

\renewcommand{\headrulewidth}{0.4pt}
\renewcommand{\footrulewidth}{0.4pt}

\newcommand{\set}[1]{\left\{ #1 \right\}}
%% We also redfine the negation symbol:
\renewcommand{\neg}{\sim}
\newtheorem{lemma}{Lemma}[section]
\theoremstyle{definition}
\newtheorem{theorem}{Theorem}[section]
\makeatletter
\newsavebox\myboxA
\newsavebox\myboxB
\newcolumntype{P}[1]{>{\centering\arraybackslash}p{#1}}
\newlength\mylenA
\newcommand*\xoverline[2][0.75]{%
    \sbox{\myboxA}{$\m@th#2$}%
    \setbox\myboxB\null% Phantom box
    \ht\myboxB=\ht\myboxA%
    \dp\myboxB=\dp\myboxA%
    \wd\myboxB=#1\wd\myboxA% Scale phantom
    \sbox\myboxB{$\m@th\overline{\copy\myboxB}$}%  Overlined phantom
    \setlength\mylenA{\the\wd\myboxA}%   calc width diff
    \addtolength\mylenA{-\the\wd\myboxB}%
    \ifdim\wd\myboxB<\wd\myboxA%
       \rlap{\hskip 0.5\mylenA\usebox\myboxB}{\usebox\myboxA}%
    \else
        \hskip -0.5\mylenA\rlap{\usebox\myboxA}{\hskip 0.5\mylenA\usebox\myboxB}%
    \fi}
\makeatother
\usepackage[]{mdframed}


% draw a frame around given text
\newcommand{\framedtext}[1]{%
\par%
\noindent\fbox{%
    \parbox{\dimexpr\linewidth-2\fboxsep-2\fboxrule}{#1}%
}%
}
\begin{document}
\centering
 \textbf{MATH 220 Practice Finals 1 --- October, 2024, Duration: 2.5 hours}
 \\
\textit{This test has \textbf{10 questions} on \textbf{20 pages}, for a total of 100 points. }
\vspace{2cm}
\renewcommand{\arraystretch}{2}
\\
\begin{tabular}{ | m{7.5cm}| m{7.5cm}| } 
  \hline
  First Name: & Last Name: \\
  \hline
  Student Number: & Section: \\
  \hline 
 \multicolumn{2}{| l |}{Signature:}  \\
  \hline
\end{tabular}
\\
\vspace{1.5cm}
\begin{tabular}{ | P{1.7cm} | P{0.4cm}| P{0.4cm}|P{0.4cm}|P{0.4cm}|P{0.4cm}|P{0.4cm}|P{0.4cm}|P{0.4cm}|P{0.4cm}|P{0.4cm}|P{0.4cm}|P{0.4cm}|P{0.4cm}|P{0.4cm}} 
  \hline
 Question: &1 & 2&3&4&5&6&7&8&9&10 \\
 \hline
 Points: & & & & & & & & & &     \\
  \hline
  Total:  & \multicolumn{10}{| r |}{/100} \\
  \hline
\end{tabular}
\clearpage
\begin{enumerate}
    \item[1.] \reversemarginpar\marginnote{ \fbox{10 Marks} }[-0.24in] Carefully define or restate each of the following: \begin{enumerate}
        \item A rational number $q \in \mathbb{Q}$
        \begin{mdframed}
            \textbf{Solution:}
            $q \in \mathbb{Q}$ if there exists coprime $a,b \in \mathbb{Z}$ where $b \neq 0$ such that $q = \frac{a}{b}$.
        \end{mdframed}
        \vspace{0.5in}
        \item Bézout's lemma
                 \begin{mdframed}
            \textbf{Solution:}
            For all $a,b \in \mathbb{Z}$, there exists $x,y \in \mathbb{Z}$ such that $ax + by = \gcd (a,b)$
        \end{mdframed}
                \vspace{0.5in}
        \item The Fundamental Theorem of Arithmetic
                 \begin{mdframed}
            \textbf{Solution:}
            Let $n \in \mathbb{N}$. Then, $n$ can be uniquely factorised into a product of prime powers ${p_1}^{e_1}{p_2}^{e_2}\ldots {p_n}^{e_n}$ where $p_i$ are distinct primes and $e_i \in \mathbb{Z}$.
        \end{mdframed}
                \vspace{0.3in}
        \item A convergent sequence $(x_n)_{n \in \mathbb{N}} : \mathbb{N} \mapsto \mathbb{R}$
                 \begin{mdframed}
            \textbf{Solution:}
             $(x_n)_{n \in \mathbb{N}} : \mathbb{N} \mapsto \mathbb{R}$ converges to $L \in \mathbb{R}$ if for all $\varepsilon > 0 \in \mathbb{R}$, there exists $N \in \mathbb{N}$ such that for all $n > N$, $|x_n - L | < \varepsilon$.
        \end{mdframed}
                \vspace{0.3in}
        \item The principle of mathematical induction
         \begin{mdframed}
            \textbf{Solution:}
            Let $\ell \in \mathbb{Z}$ and let $S = \set{k \in \mathbb{Z} | n \geq \ell}$. If $P(\ell)$ is true and $P(k)$ being true implies $P(k+1)$ being true for some $k \in S$, then $P(n)$ is true for all $n \in S$.
        \end{mdframed}
    \end{enumerate}
\end{enumerate}
\pagebreak
\begin{enumerate}
    \item[2.] \reversemarginpar\marginnote{ \fbox{10 Marks} }[-0.24in] Write the negation of each of the following and prove or disprove the original statement.
    \begin{enumerate}
        \item For all $x \in \mathbb{R}$, there exists $y \in \mathbb{R}$ such that for all $z \in \mathbb{R}$, if $x + y < z$, then $x -y > z$.
    \end{enumerate}
     \begin{mdframed}
            \textbf{Solution:} The negation is "There exists $x \in \mathbb{R}$ such that for all $y \in \mathbb{R}$, there exists $z \in \mathbb{R}$ such that $x+y < z$ and $x-y \leq z$". The original statement is false. \renewcommand{\proofname}{Disproof} \begin{proof}
                Let $x = 0$ and $y \in \mathbb{R}$. Choose $z = |y| + 1$. Notice $\pm y \leq |y| <  1 + |y|$, so $z > x+y$ and $z > x-y$ so the statement is false. 
            \end{proof}
        \end{mdframed}
\end{enumerate}
\pagebreak
\begin{enumerate}
    \item[] \begin{enumerate}
        \item[(b)] There exists $x \in \mathbb{R}$ such that for all $y \in \mathbb{R}$, for all $z \in \mathbb{R}$, $xy > z$.
        
    \end{enumerate}
     \begin{mdframed}
            \textbf{Solution:} The negation is "For all $x \in \mathbb{R}$, there exists $y \in \mathbb{R}$ such that there exists $z \in \mathbb{R}$ such that $xy \leq z$". The original statement is false. \renewcommand{\proofname}{Disproof} \begin{proof}
                Let $x \in \mathbb{R}$ and choose $y = z = 0$. Then, $xy = 0 = z$ so the statement is false. 
            \end{proof}
        \end{mdframed}
\end{enumerate}
\pagebreak
\begin{enumerate}
    \item[3. ]\reversemarginpar\marginnote{ \fbox{10 Marks} }[-0.24in] Let $f: A \mapsto B$ and $g: B \mapsto C$ be functions. Prove or disprove each of the following: \begin{enumerate}
        \item For all $U \subseteq C$, $(g \circ f)^{-1} (U) = f^{-1}(g^{-1}(U))$.
        
    \end{enumerate}
     \begin{mdframed}
            \textbf{Solution:} 
           \begin{proof}
              Let everything be as stated. We show each inclusion in turn. \begin{itemize}
                  \item Assume $x \notin f^{-1}(g^{-1}(U))$, so $f(x) \notin g^{-1}(U)$ and $g(f (x)) \notin U$. It follows that $x \notin (g \circ f)^{-1} (U)$, so by contraposition, $(g \circ f)^{-1} (U) \subseteq f^{-1}(g^{-1}(U))$
                  \item Assume $x \in f^{-1}(g^{-1}(U))$. Then, $f(x) \in g^{-1}(U)$ and $g(f(x)) \in U$ so $(g \circ f)^{-1} (U) \subseteq f^{-1}(g^{-1}(U))$.
              \end{itemize}
              It follows that $(g \circ f)^{-1} (U) = f^{-1}(g^{-1}(U))$. as required.
           \end{proof} 
        \end{mdframed}
\end{enumerate}
\pagebreak
\begin{enumerate}
    \item[] \begin{enumerate}
        \item[(b)] For all $U \subseteq B$, $(g\circ f)^{-1}(g(U)) = f^{-1}((U))$
    \end{enumerate}
     \begin{mdframed}
            \textbf{Solution:}
            \renewcommand{\proofname}{Disproof} \begin{proof}
                Let $A = \set{1} = C$, $B = \set{0,1}$, $f(x) = 1$ and $g(x) = 1$. Then, notice for $U = \set{0}$, $f^{-1}((U)) = \varnothing$, but $(g\circ f)^{-1}g(U) = (g\circ f)^{-1} (\set{1}) = \set{1}$ so $(g\circ f)^{-1}(g(U)) \neq f^{-1}((U))$ and thus the statement is false.
            \end{proof}
        \end{mdframed}
\end{enumerate}
\pagebreak
\begin{enumerate}
    \item[4.]\reversemarginpar\marginnote{ \fbox{10 Marks} }[-0.24in] Let $n \in \mathbb{N}$ be even and $\mathbb{Z} / n\mathbb{Z} = \set{0,1,2,\ldots,n-1}$. Let $S = \set{k \in \mathbb{Z} / n\mathbb{Z}  | 2k \equiv 0 \mod n }$. Prove that $|S|$ is even.
     \begin{mdframed}
            \textbf{Solution:}
            \begin{proof}
                Assume for the sake of contradiction that $|S|$ is not even, so for there exists an odd number of non-zero elements $k \in \mathbb{Z}/n\mathbb{Z}$ such that $2k \equiv 0 \mod n$. Consider $\mathbb{Z} / n\mathbb{Z}-S$, which has an odd number of elements. Let $a \in \mathbb{Z} / n\mathbb{Z}-S$. Since $2a \not \equiv 0 \mod n$, it follows that there exists $a^{-1} \in \mathbb{Z} / n\mathbb{Z}-S$ such that $a + a^{-1} \equiv 0 \mod n$, but there are only an odd number of $a$ so there exists $a_0 \in \mathbb{Z} / n\mathbb{Z}-S $ such that $a_0 = {a_0}^{-1}$ which implies $2a_0 \equiv 0 \mod n$, a contradiction. Hence, $|S|$ is even as required.
            \end{proof}
        \end{mdframed}
\end{enumerate}
\pagebreak
\begin{enumerate}
    \item[5.] \begin{enumerate} \item \reversemarginpar\marginnote{ \fbox{10 Marks} }[-0.24in] Prove that $f: \mathbb{R} \mapsto \mathbb{C}\backslash \set{0}$ where $f(x) = e^{2\pi i x}$ is neither injective nor surjective and for all $x,y \in \mathbb{R}$, $f(x+y) = f(x)f(y)$. \end{enumerate}
     \begin{mdframed}
            \textbf{Solution:}
            \begin{proof}
                Notice $0 \neq 1$ but $f(0) = f(1) = 0 \in \mathbb{C}$ so $f$ is not injective. Also notice there does not exist $x \in \mathbb{R}$ such that $f(x) = 0$ since $e^{2\pi i x} \neq 0$ for all $x \in \mathbb{R}$. Hence, $f$ is not surjective. Finally, let $x,y \in \mathbb{R}$. Observe that $f(x+y) = e^{2\pi i (x+y)} = e^{2\pi i x + 2\pi i y} = e^{2\pi i x} e^{2\pi i y} = f(x)f(y)$ so $f(x+y) = f(x)f(y)$ as required.
            \end{proof}
        \end{mdframed}
\end{enumerate}
\pagebreak
\begin{enumerate}
    \item[] \begin{enumerate}
        \item[(b)] Let $R$ be a relation on $\mathbb{R}$ be defined as $x R y$ if and only if $x = y + z$ where $z \in f^{-1}(\set{1})$. Prove that $R$ is an equivalence relation.
    \end{enumerate}
     \begin{mdframed}
            \textbf{Solution:}
            \begin{proof} We prove reflexivity, symmetry, and transitivity in turn. \begin{itemize}
                \item Notice $x = x + 0$ and $0 \in f^{-1}(\set{1})$ so $x R x$ as required.
                \item Assume $x R y$. Then, notice $y = x - z$ and since $f(-z) = e^{-2\pi i z} = \frac{1}{f(z)} = 1$, we have $-z \in f^{-1}(\set{1})$ and so $y R x$.
                \item Assume $x R y$ and $y R w$. Then, $x = y + z_1$ and $y = w + z_2$ so $x = w + z_2 + z_1$. Now notice $f(z_2+z_1) = f(z_2)f(z_1) = 1 \cdot 1 = 1$ so $z_2 + z_1 \in f^{-1}(\set{1})$ and thus $x R w$
            \end{itemize}
            It follows that $R$ is an equivalence relation. \end{proof}
        \end{mdframed}
\end{enumerate}
\pagebreak
\begin{enumerate}
    \item[] \begin{enumerate}
        \item[(c)] Find all equivalence classes under $R$. Show that the operation $[a] + [b] = [a+b]$ is well defined for $a,b \in \mathbb{R}$. 
    \end{enumerate}
     \begin{mdframed}
            \textbf{Solution:}
            \begin{proof}
            There are infinite equivalence classes. In particular, for all $x \in [0,1)$, there exists an equivalence class of the form $[x] = \set{k + x | k \in \mathbb{Z}}$. For showing the operation is well defined, we claim that $x Ry$ if and only if $f(x) = f(y)$. Assume $x = y + z$ where $z \in f^{-1}(\set{1})$. Then, $f(x) = f(y+z) = f(y)f(z) = f(y)$ so $f(y) = f(x)$. On the other hand, assume $f(x) = f(y)$. Notice for all $k \in \mathbb{Z}$, $f(k) = 1$. Then, $f(x)f(z_1) = f(x)1 = f(y)1 = f(y)f(z_2) = f(y+z_2)$ where $z_1,z_2 \in f^{-1}(\set{1})$ so $x = y + z_2 - z_1$ and since $z_2-z_1 \in f^{-1}(\set{1})$, $x R y$. Let $a, b \in \mathbb{R}$. Since for all $y \in [a+b]$, $f(y) = f(a+b)$, it suffices to show that for all $x \in [a] + [b], f(x) = f(a+b)$. Notice since $x \in [a] + [b]$, $x = a + z_1 + b + z_2$ where $z_1,z_2 \in f^{-1}(\set{1})$ so $f(x) = f(a+z_1+b+z_2) = f(a)f(z_1)f(b)f(z_2) = f(a)f(b) = f(a+b)$ so $x \in [a+b]$. Hence, the operation is well defined.
            
        \end{proof}
        \end{mdframed}
\end{enumerate}
\pagebreak
\begin{enumerate}
    \item[] \begin{enumerate}
        \item[(d)] Let $\mathbb{R}/R$ denotes the set of equivalence classes under $R$. Find a bijective map $g : \mathbb{R}/R \mapsto \mathrm{Im}(f)$ such that $g([x] + [y]) = g([x])g([y])$. Prove your result.
    \end{enumerate}
     \begin{mdframed}
            \textbf{Solution:}
            \begin{proof}
                We claim the map $g([x]) = e^{2\pi i x}$ is such bijective map. \begin{itemize}
                    \item For injectivity, assume $g([x]) = g([y])$, so $ e^{2\pi i x} =  e^{2\pi i y}$. It follows that $f(x) = f(y)$ so $[x] = [y]$.
                    \item For surjectivity, let $z \in \mathrm{Im}(f)$, so there exists $x \in \mathbb{R}$ such that $z = f(x)$. It follows that $g([x]) = e^{2\pi i x} = f(x) = z$ so $[x] \in g^{-1}(\set{z})$
                \end{itemize}
                Finally, notice $g([x] + [y]) = e^{2\pi (x + y)} = e^{2\pi i x + 2\pi i y} = e^{2\pi i x} e^{2\pi i y} = g([x])g([y])$ so $g([x] + [y]) = g([x])g([y])$ and thus $g$ is such a map as required.
            \end{proof}
        \end{mdframed}
\end{enumerate}
\pagebreak
\begin{enumerate}
    \item[6.]  \begin{enumerate}
    \item \reversemarginpar\marginnote{ \fbox{10 Marks} }[-0.24in] 
Let $X$ be a non-empty set. Prove that any equivalence relation on $X$ forms a partition on $X$.
\end{enumerate}
 \begin{mdframed}
            \textbf{Solution:}
            \begin{proof}
               Let $\sim$ be an equivalence relation on $X$. By reflexivity, we have that for all $x \in X$, $x$ lies in some equivalence class and thus every equivalence class is non-empty. For showing that equivalence classes are disjoint, suppose $x \in [a]$ and $x \in [b]$, so $x \sim a$ and $x \sim b$. By symmetry and transitivity, we have $a \sim x$ and $x \sim b$ so $[a] = [b]$. Thus, equivalence classes are disjoint. It follows that the equivalence classes of an equivalence relation partition a set.
            \end{proof}
        \end{mdframed}
\end{enumerate}
\pagebreak
\begin{enumerate}
    \item[] \begin{enumerate}
        \item[(b)] Prove that any partition on $X$ corresponds to equivalence classes of an equivalence relation on $X$.
    \end{enumerate}
     \begin{mdframed}
            \textbf{Solution:}
            \begin{proof}
                Let $P \subseteq \mathcal{P}(X)$ be a partition of $X$, and define a relation $x \sim y$ if and only if $x,y \in A_i$ where $A_i \in P$. \begin{itemize}
                    \item Since every $x$ lies in some $A_i \in P$ by the definition of a partition, $x \sim x$.
                    \item If $x, y \in A_i$, then surely $y,x \in A_i$ so $x \sim y$.
                    \item Similarly, if $x ,y \in A_i$ and $y,z \in A_i$, it follows that $x , z \in A_i$ since every element lies in precisely one $A_i$.
                \end{itemize}
                Hence, $\sim$ is an equivalence relation and by our choice of equivalence relation, we have that every equivalence class corresponds precisely to some $A_i \in P$.
            \end{proof}
        \end{mdframed}
\end{enumerate}
\pagebreak
\begin{enumerate}
    \item[7.] \reversemarginpar\marginnote{ \fbox{10 Marks} }[-0.24in] Prove that if $a \neq 0$, $\lim_{x \rightarrow a} \frac{1}{x} = \frac{1}{a}$.
     \begin{mdframed}
            \textbf{Solution:}
            \begin{proof}
            Notice $a > 0$, or $a<0$.
            \begin{itemize}
                \item Let $\varepsilon > 0$ and assume $a>0$. Choose $\delta = \min(\set{b,(a^2-a)\varepsilon})$ where $0 < b < a$. Assume $0 < |x-a| < \delta$, so $|x-a| < b$ which implies $a-b < |x|$. Then, notice $\left | \frac{1}{x} - \frac{1}{a}\right | = \left |\frac{a-x}{ax} \right | < \frac{\delta}{a^2-ab} = \varepsilon$ so $\lim_{x \rightarrow a} \frac{1}{x} = \frac{1}{a}$ as required.
                \item Let $\varepsilon > 0$ and assume $a<0$. Choose $\delta = \min(\set{-b,|a^2 - ba|\varepsilon})$ where $0 < -b < -a$. Assume $0 < |x-a| < \delta$, so $|x-a| < -b$ which implies $|a-b| < |x|$. Then, notice $\left | \frac{1}{x} - \frac{1}{a}\right | = \left |\frac{a-x}{ax} \right | < \frac{\delta}{|a^2 - ba|} = \varepsilon$ so $\lim_{x \rightarrow a} \frac{1}{x} = \frac{1}{a}$ as required.
            \end{itemize}
            \end{proof}
        \end{mdframed}
\end{enumerate}
\pagebreak
\begin{enumerate}
    \item[8.] \reversemarginpar\marginnote{ \fbox{10 Marks} }[-0.24in] Prove or disprove each of the following: \begin{enumerate}
        \item Let $A,B \subseteq C$. If $|C \backslash A| = |C \backslash B \backslash C|$, then $|A| = |B|$.
    \end{enumerate}
     \begin{mdframed}
            \textbf{Solution:}
            \renewcommand{\proofname}{Disproof}
            \begin{proof}
                 Let $A  = \set{1}$, $B = \set{1,2}$ and $C = \mathbb{N}$. Notice there exists an explicit bijection $f: C \backslash A \mapsto  C \backslash B$ where $f(n) = n+1$, but $|A| = 1 \neq 2 = |B|$ so the statement is false.
            \end{proof}
        \end{mdframed}
\end{enumerate}
\pagebreak
\begin{enumerate}
    \item[] \begin{enumerate}
        \item[(b)] Let $A_i \in X$. Then, $X \backslash (\bigcup_{i \in I} A_i)   = \bigcap_{i \in I} (X \backslash A_i )$.
    \end{enumerate}
     \begin{mdframed}
            \textbf{Solution:}
            \begin{proof}
                Let $A_i \in X$ and $x \in X$. We show each inclusion in turn.
                \begin{itemize}
                    \item Let $x \in X \backslash (\bigcup_{i \in I} A_i)$, so $x \notin (\bigcup_{i \in I} A_i)$ and thus $x \notin A_i$ for all $i \in I$. It follows that $x \in A_i \backslash X$ for all $i$ and thus $x \in  \bigcap_{i \in I} (X \backslash A_i)$.
                    \item Let $x \in  \bigcap_{i \in I} (X \backslash A_i)$ so for all $i \in I$, $x \notin A_i$. It follows that $x \notin \bigcup_{i \in I} A_i$ so $x \in X \backslash (\bigcup_{i \in I} A_i)$.
                \end{itemize}$X \backslash (\bigcup_{i \in I} A_i)   = \bigcap_{i \in I} (X \backslash A_i )$.
                Hence, 
            \end{proof}
        \end{mdframed}
\end{enumerate}
\pagebreak
\begin{enumerate}
    \item[9.] \reversemarginpar\marginnote{ \fbox{10 Marks} }[-0.24in] 
 Let $B_r(x) \in \mathbb{R}^n$, called an open ball of radius $r$, be defined as $\set{y \in \mathbb{R}^n | \| x-y \| < r}$ for some $r > 0$, note that $\| x-y \|$ refers to the Euclidean norm $\| x-y \| = \sqrt{\sum_{i=1}^n (x_i - y_i)^2}$. \begin{enumerate}
        \item Let $p \in \mathbb{R}^n$ and $q \in B_r(p)$. Show that there exists $B_{r_2}(q) \subseteq B_r(p)$. (Hint: The triangle inequality works for $\|x - y\| \leq \|x - z\| + \|z -y \|$ in $\mathbb{R}^n$ too)
    \end{enumerate}
      \begin{mdframed}
            \textbf{Solution:}
            \begin{proof}
                Let $q \in B_r(p)$. Choose $r_2 = r - \| p-q\|$ and $q \in B_{r_2}(q)$. Now, $r > \| p-q\|$ so $r_2 > 0$. Let $x \in B_{r_2}(q)$. By the triangle inequality, $\|p-x \| \leq \|p-q\| + \|q-x\| < \|p-q\| + r_2 = r$ so $B_{r_2}(q) \subseteq B_r(p)$. 
            \end{proof}
        \end{mdframed}
\end{enumerate}
\pagebreak
\begin{enumerate}
    \item[] \begin{enumerate}
        \item[(b)] A set $E \subseteq \mathbb{R}^n$ is open if for all $p \in E$, there exists $B_r(p) \subseteq E$. Prove that $E$ is open if and only if $E$ is a union of open balls.
    \end{enumerate}
      \begin{mdframed}
            \textbf{Solution:}
            \begin{proof}
               We prove each direction in turn. \begin{itemize}
                    \item For one direction, suppose $E$ is open. I claim $E = \bigcup_{x \in E} B_r(x)$ where $B_r(x)$ is some open ball centered at $x$ contained in $E$. Notice this yields $\bigcup_{x \in E} B_r(x) \subseteq E$. On the other hand, if $x \in E$, we know $x \in B_r(x) \subseteq E$ for some $r > 0$ by $E$ is open, so $x \in \cup_{x \in E} B_r(x)$. This proves $E = \cup_{x \in E} B_r(x)$ so $E$ is a union of open balls.
                    \item For the other direction, suppose $E$ is a union of open balls $\set{B_i}$, and let $x \in E$. By definition, $x \in {B_i}_0$ for some ${B_i}_0 \in \set{B_i}$ and from part (a), we know there exists $B_r(x) \subseteq {B_i}_0 \subseteq E$ so $E$ is open as required.
                \end{itemize}
            \end{proof}
        \end{mdframed}
\end{enumerate}
\pagebreak
\begin{enumerate}
    \item[10.] \reversemarginpar\marginnote{ \fbox{10 Marks} }[-0.24in] 
Let $N= \set{1,2,3,\ldots,n}$ for some $n \in \mathbb{N}$, and let $S_n$ be the set of bijective functions $f: N \mapsto N$. \begin{enumerate}
    \item Prove by induction that for all $n \in \mathbb{N}$, $|S_n|=n!$.
\end{enumerate}
 \begin{mdframed}
            \textbf{Solution:}
            \begin{proof}
               Let $N= \set{1,2,3,\ldots,n}$ for some $n \in \mathbb{N}$, and let $S_n$ be the set of bijective functions $f: N \mapsto N$. We proceed with mathematical induction on $n$. \begin{itemize}
                   \item For the base case, notice when $n = 1$, the only bijective map is $f(n) = 1$ so $S_n = \set{f}$ and $|S_n| = 1$.
                   \item Assume that for $n = k$, $|S_k| = k$. For $|S_{k+1}|$, notice taking every map in $S_k$ and fixing an element yields $k!$ bijections and hence elements in $S_{k+1}$, and fixing and repeating this process with a different element in $N = \set{1,2,3,\ldots,k+1}$ will give $(k+1)k!$ possible bijections from $N$ to itself and hence $(k+1)!$ bijections. It follows that $|S_{k+1} | = (k+1)!$ as required.
               \end{itemize} 
               By the principle of mathematical induction, $|S_n| = n!$ for all $n \in \mathbb{N}$.
            \end{proof}
        \end{mdframed}
\end{enumerate}
\pagebreak
\begin{enumerate}
   \item[] \begin{enumerate}
       \item[(b)]  Prove by induction that for all $n \geq 3 \in \mathbb{N}$, there exists $f,g \in S_n$ such that $f \circ g \neq g \circ f$.
   \end{enumerate}
    \begin{mdframed}
            \textbf{Solution:}
            \begin{proof}
            Let $n \geq 3$. We proceed with mathematical induction on $n$. \begin{itemize}
                \item For the base case, let $f$ and $g$ in $S_3$ be defined as follows: $$f(n) = \left \{\begin{matrix}
                    1 \mapsto 2 \\ 2 \mapsto 1 \\ 3 \mapsto 3
                \end{matrix} \right. , g(n) = \left \{\begin{matrix}
                    1 \mapsto 2 \\ 2 \mapsto 3 \\ 3 \mapsto 1
                \end{matrix} \right.$$ Notice $f\circ g(1) = 1$ and $g \circ f(1) = 3$ so $f \circ g \neq g \circ f$.
                \item Assume there exists $f,g \in S_k$ such that $f \circ g \neq g \circ f$ for some $k$. For $S_{k+1}$, choose such $f,g \in S_k$ and create $f',g'$ which is the same as $f$ and $g$ except that the $k+1$th element gets mapped to itself. Notice there exists some $i \in \set{1,2,\ldots k}$ such that $f'\circ g'(i) = f \circ g(i) \neq g \circ f (i)= g' \circ f' (i)$ so it follows that $f' \circ g' \neq g'\circ f'$ and thus the result holds.
            \end{itemize}
              By the principle of mathematical induction, the result holds for all $n \geq 3$.
            \end{proof}
        \end{mdframed}
\end{enumerate}
\end{document}
