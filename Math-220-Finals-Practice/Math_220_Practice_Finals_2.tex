\documentclass[letterpaper,12pt]{article}
\newcommand{\hw}{2} 
\usepackage{amsmath, amsfonts, amssymb, amsthm}
\usepackage[paper=letterpaper,left=25mm,right=25mm,top=3cm,bottom=25mm]{geometry}
\usepackage{fancyhdr} %% for details on how this work, search-engine ``fancyhdr documentation''
\pagestyle{fancy}
\usepackage{array}
\usepackage{marginnote}
\lhead{MATH 220 Practice Finals 2} % course name as top-left
\chead{Page \thepage \ of 19} % homework number in top-centre
\rhead{Student No: \ \ \ \ \ \ \ \ \ \ \ \ \ \ \ \ \ \ \ \ }

\cfoot{Page \thepage \ of 19} % page in middle
\usepackage{ragged2e}

\renewcommand{\headrulewidth}{0.4pt}
\renewcommand{\footrulewidth}{0.4pt}

\newcommand{\set}[1]{\left\{ #1 \right\}}
%% We also redfine the negation symbol:
\renewcommand{\neg}{\sim}
\newtheorem{lemma}{Lemma}[section]
\theoremstyle{definition}
\newtheorem{theorem}{Theorem}[section]
\makeatletter
\newsavebox\myboxA
\newsavebox\myboxB
\newcolumntype{P}[1]{>{\centering\arraybackslash}p{#1}}
\newlength\mylenA
\newcommand*\xoverline[2][0.75]{%
    \sbox{\myboxA}{$\m@th#2$}%
    \setbox\myboxB\null% Phantom box
    \ht\myboxB=\ht\myboxA%
    \dp\myboxB=\dp\myboxA%
    \wd\myboxB=#1\wd\myboxA% Scale phantom
    \sbox\myboxB{$\m@th\overline{\copy\myboxB}$}%  Overlined phantom
    \setlength\mylenA{\the\wd\myboxA}%   calc width diff
    \addtolength\mylenA{-\the\wd\myboxB}%
    \ifdim\wd\myboxB<\wd\myboxA%
       \rlap{\hskip 0.5\mylenA\usebox\myboxB}{\usebox\myboxA}%
    \else
        \hskip -0.5\mylenA\rlap{\usebox\myboxA}{\hskip 0.5\mylenA\usebox\myboxB}%
    \fi}
\makeatother
\begin{document}
\centering
 \textbf{MATH 220 Practice Finals 2 --- October, 2024, Duration: 2.5 hours}
 \\
\textit{This test has \textbf{9 questions} on \textbf{19 pages}, for a total of 90 points. }
\vspace{2cm}
\renewcommand{\arraystretch}{2}
\\
\begin{tabular}{ | m{7.5cm}| m{7.5cm}| } 
  \hline
  First Name: & Last Name: \\
  \hline
  Student Number: & Section: \\
  \hline 
 \multicolumn{2}{| l |}{Signature:}  \\
  \hline
\end{tabular}
\\
\vspace{1.5cm}
\begin{tabular}{ | P{1.7cm} | P{0.4cm}| P{0.4cm}|P{0.4cm}|P{0.4cm}|P{0.4cm}|P{0.4cm}|P{0.4cm}|P{0.4cm}|P{0.4cm}|P{0.4cm}|P{0.4cm}|P{0.4cm}|P{0.4cm}|P{0.4cm}} 
  \hline
 Question: &1 & 2&3&4&5&6&7&8&9 \\
 \hline
 Points: & & & & & & & & &     \\
  \hline
  Total:  & \multicolumn{9}{| r |}{/90} \\
  \hline
\end{tabular}
\clearpage
\begin{enumerate}
    \item[1.] \reversemarginpar\marginnote{ \fbox{10 Marks} }[-0.24in] Carefully define or restate each of the following: \begin{enumerate}
        \item A relation on $A$
        \vspace{1.4in}
        \item A function $f:A \mapsto B$
                \vspace{1.4in}
        \item The limit of a function $f: \mathbb{R} \mapsto \mathbb{R}$ as $x \rightarrow a$
                \vspace{1.4in}
        \item Euclidean division
                \vspace{1.4in}
        \item The principle of mathematical induction
    \end{enumerate}
\end{enumerate}
\pagebreak
\begin{enumerate}
    \item[2.] \reversemarginpar\marginnote{ \fbox{10 Marks} }[-0.24in] Write the negation of each of the following and prove or disprove the original statement.
    \begin{enumerate}
        \item There exists a prime $p \in \mathbb{Z}$ such that there exists $r,k \in \mathbb{Z}$ such that $0 \leq r \leq p-1$ and $k < p$ and $p \mid kr$.
    \end{enumerate}
\end{enumerate}
\pagebreak
\begin{enumerate}
    \item[] \begin{enumerate}
        \item[(b)] For all $n \in \mathbb{N}$, for all $p,q \in \mathbb{N}$ such that $k,\ell \in \mathbb{N}$ are the smallest integers such that $n \mid kp$ and $n \mid \ell q$, then if $k = \ell$, $|\set{ px \in \mathbb{Z} | 0 \leq px \leq kp \in \mathbb{Z}}| = |\set{ qy \in \mathbb{Z} | 0 \leq y q \leq \ell q \in \mathbb{Z}}|$
    \end{enumerate}
\end{enumerate}
\pagebreak
\begin{enumerate}
    \item[3. ]\reversemarginpar\marginnote{ \fbox{10 Marks} }[-0.24in] Let $(\mathbb{Q},+)$ be the set of rational numbers with addition as the operation and $(\mathbb{Q}\backslash \set{0},\cdot)$ be the set of rational numbers except 0 with multiplication as the operation. Prove that there does not exist a bijective $f: (\mathbb{Q},+) \mapsto (\mathbb{Q}\backslash \set{0},\cdot)$ such that for all $x,y \in (\mathbb{Q},+)$, $f(x+y) = f(x)f(y)$.
\end{enumerate}
\pagebreak
\begin{enumerate}
    \item[4.]\reversemarginpar\marginnote{ \fbox{10 Marks} }[-0.24in] Let $S = \set{1,2,3,\ldots,m \in \mathbb{N}}$ and $p_i:S \mapsto S$ be a function that swaps some $a,b \in S$ around where $a \neq b$ and fixes everything else, denoted $p_i = (ab)$. Prove that the identity map $e$ can only be written as a product $p_1p_2p_3\ldots p_n$ where $n$ is even by induction on the number of $p_i$.
\end{enumerate}
\pagebreak
\begin{enumerate}
    \item[5.]  \begin{enumerate}
    \item \reversemarginpar\marginnote{ \fbox{10 Marks} }[-0.24in] 
Let $D_n$ denote the set of rotations by an angle of $\frac{2\pi}{i}$ where $1 \leq n$ and reflections across an axis of symmetry on an $n-$gon, and let $\sim$ be a relation $\sigma \sim \tau$ if and only if $\sigma \tau$ is a rotation where $\sigma, \tau \in D_n$. Prove that $R$ is an equivalence relation.
\end{enumerate}
\end{enumerate}
\pagebreak
\begin{enumerate}
    \item[] \begin{enumerate}
        \item[(b)] Let $R$ be the set of rotations and for all $\sigma \in D_n$, let $\sigma R$ denote the set $\set{\sigma x | x \in R}$. Prove that the collection of $\sigma R$ forms a partition on $D_n$. Furthermore, prove that for all $\sigma \in D_n$, $\sigma R = [\sigma]$ where $[\sigma]$ is the equivalence class of $\sigma$ under $\sim$.
    \end{enumerate}
\end{enumerate}
\pagebreak
\begin{enumerate}
    \item[6.] \begin{enumerate}
        \item \reversemarginpar\marginnote{ \fbox{10 Marks} }[-0.24in] Prove that if $A_1,A_2,A_3 \ldots A_n$ and $B_1,B_2,B_3,\ldots B_n$ are non-empty sets such that for all $i \leq n \in \mathbb{N}$, $|A_i| \leq |B_i|$, then $|\prod_{i = 1}^n A_i |\leq |\prod_{i = 1}^n B_i |$ by constructing an explicit injection $f: \prod_{i = 1}^n A_i  \mapsto \prod_{i = 1}^n B_i $
    \end{enumerate}
\end{enumerate}
\pagebreak
\begin{enumerate}
    \item[] \begin{enumerate}
        \item[(b)] Prove that the result no longer holds when there exists $B_i$ such that $B_i$ is empty.
    \end{enumerate}
\end{enumerate}
\pagebreak
\begin{enumerate}
    \item[7.] \reversemarginpar\marginnote{ \fbox{10 Marks} }[-0.24in] Prove or disprove each of the following: \begin{enumerate}
        \item There exists a bijective function $f:\mathbb{R} \mapsto S_1$ where $S_1$ is the unit circle in $\mathbb{C}$.
    \end{enumerate}
\end{enumerate}
\pagebreak
\begin{enumerate}
    \item[] \begin{enumerate}
        \item[(b)] For all $A,B$, $|A \times B| = |B \times A|$.
    \end{enumerate}
\end{enumerate}
\pagebreak
\begin{enumerate}
    \item[] \begin{enumerate}
        \item[(c)] A countable union of countable sets is countable.
    \end{enumerate}
\end{enumerate}
\pagebreak
\begin{enumerate}
    \item[] \begin{enumerate}
        \item[(d)] A denumerable intersection of uncountable sets is countable.
    \end{enumerate}
\end{enumerate}
\pagebreak
\begin{enumerate}
    \item[8.] \reversemarginpar\marginnote{ \fbox{10 Marks} }[-0.24in] 
 Let $B_r(x) \in \mathbb{R}^n$, called an open ball of radius $r$, be defined as $\set{y \in \mathbb{R}^n | \| x-y \| < r}$ for some $r > 0$, note that $\| x-y \|$ refers to the Euclidean norm $\| x-y \| = \sqrt{\sum_{i=1}^n (x_i - y_i)^2}$. \begin{enumerate}
        \item Let $p \in \mathbb{R}^n$ and $q \in B_r(p)$. Show that there exists $B_{r_2}(q) \subseteq B_r(p)$.
    \end{enumerate}
\end{enumerate}
\pagebreak
\begin{enumerate}
    \item[] \begin{enumerate}
        \item[(b)] A set $E \subseteq \mathbb{R}^n$ is open if for all $p \in E$, there exists $B_r(p) \subseteq E$. Prove that $E$ is open if and only if $E$ is a union of open balls.
    \end{enumerate}
\end{enumerate}
\pagebreak
\begin{enumerate}
    \item[9.] \begin{enumerate}
    \item\reversemarginpar\marginnote{ \fbox{10 Marks} }[-0.24in]   Let $A \subseteq \mathbb{R}$. Prove that if $-A = \set{-a | a \in A}$, then $-\sup A = \inf A$. 
\end{enumerate}
\end{enumerate}
\pagebreak
\begin{enumerate}
   \item[] \begin{enumerate}
       \item[(b)] Let $(x_n)_{n\in\mathbb{N}} : \mathbb{N} \mapsto \mathbb{R}$ be a non-decreasing sequence, and let $X= \set{x_n | n \in \mathbb{N}}$. Prove that if $(x_n)_{n\in\mathbb{N}}$ is bounded above, then $\lim_{n \rightarrow \infty} x_n = \sup X$. Likewise, prove that if $(x_n)_{n\in\mathbb{N}}$ is a non-increasing sequence, and if $(x_n)_{n\in\mathbb{N}}$ is bounded below, then $\lim_{n \rightarrow \infty} x_n = \inf X$
   \end{enumerate}
\end{enumerate}
\pagebreak
\begin{enumerate}
    \item[] \begin{enumerate}
        \item[(c)] Prove that if $(x_n)_{n\in\mathbb{N}} : \mathbb{N} \mapsto \mathbb{R}$ is bounded, then $(x_n)_{n \in \mathbb{N}} : \mathbb{N} \mapsto \mathbb{R}$ has a finite limit.
    \end{enumerate}
\end{enumerate}
\end{document}
