\documentclass[letterpaper,12pt]{article}
\newcommand{\hw}{2} 
\usepackage{amsmath, amsfonts, amssymb, amsthm}
\usepackage[paper=letterpaper,left=25mm,right=25mm,top=3cm,bottom=25mm]{geometry}
\usepackage{fancyhdr} %% for details on how this work, search-engine ``fancyhdr documentation''
\pagestyle{fancy}
\usepackage{array}
\usepackage{marginnote}
\lhead{MATH 220 Practice Finals 1} % course name as top-left
\chead{Page \thepage \ of 20} % homework number in top-centre
\rhead{Student No: \ \ \ \ \ \ \ \ \ \ \ \ \ \ \ \ \ \ \ \ }

\cfoot{Page \thepage \ of 20} % page in middle
\usepackage{ragged2e}

\renewcommand{\headrulewidth}{0.4pt}
\renewcommand{\footrulewidth}{0.4pt}

\newcommand{\set}[1]{\left\{ #1 \right\}}
%% We also redfine the negation symbol:
\renewcommand{\neg}{\sim}
\newtheorem{lemma}{Lemma}[section]
\theoremstyle{definition}
\newtheorem{theorem}{Theorem}[section]
\makeatletter
\newsavebox\myboxA
\newsavebox\myboxB
\newcolumntype{P}[1]{>{\centering\arraybackslash}p{#1}}
\newlength\mylenA
\newcommand*\xoverline[2][0.75]{%
    \sbox{\myboxA}{$\m@th#2$}%
    \setbox\myboxB\null% Phantom box
    \ht\myboxB=\ht\myboxA%
    \dp\myboxB=\dp\myboxA%
    \wd\myboxB=#1\wd\myboxA% Scale phantom
    \sbox\myboxB{$\m@th\overline{\copy\myboxB}$}%  Overlined phantom
    \setlength\mylenA{\the\wd\myboxA}%   calc width diff
    \addtolength\mylenA{-\the\wd\myboxB}%
    \ifdim\wd\myboxB<\wd\myboxA%
       \rlap{\hskip 0.5\mylenA\usebox\myboxB}{\usebox\myboxA}%
    \else
        \hskip -0.5\mylenA\rlap{\usebox\myboxA}{\hskip 0.5\mylenA\usebox\myboxB}%
    \fi}
\makeatother
\begin{document}
\centering
 \textbf{MATH 220 Practice Finals 1 --- October, 2024, Duration: 2.5 hours}
 \\
\textit{This test has \textbf{10 questions} on \textbf{20 pages}, for a total of 100 points. }
\vspace{2cm}
\renewcommand{\arraystretch}{2}
\\
\begin{tabular}{ | m{7.5cm}| m{7.5cm}| } 
  \hline
  First Name: & Last Name: \\
  \hline
  Student Number: & Section: \\
  \hline 
 \multicolumn{2}{| l |}{Signature:}  \\
  \hline
\end{tabular}
\\
\vspace{1.5cm}
\begin{tabular}{ | P{1.7cm} | P{0.4cm}| P{0.4cm}|P{0.4cm}|P{0.4cm}|P{0.4cm}|P{0.4cm}|P{0.4cm}|P{0.4cm}|P{0.4cm}|P{0.4cm}|P{0.4cm}|P{0.4cm}|P{0.4cm}|P{0.4cm}} 
  \hline
 Question: &1 & 2&3&4&5&6&7&8&9&10 \\
 \hline
 Points: & & & & & & & & & &     \\
  \hline
  Total:  & \multicolumn{10}{| r |}{/100} \\
  \hline
\end{tabular}
\clearpage
\begin{enumerate}
    \item[1.] \reversemarginpar\marginnote{ \fbox{10 Marks} }[-0.24in] Carefully define or restate each of the following: \begin{enumerate}
        \item A rational number $q \in \mathbb{Q}$
        \vspace{1.4in}
        \item Bézout's lemma
                \vspace{1.4in}
        \item The Fundamental Theorem of Arithmetic
                \vspace{1.4in}
        \item A convergent sequence $(x_n)_{n \in \mathbb{N}} : \mathbb{N} \mapsto \mathbb{R}$
                \vspace{1.4in}
        \item The principle of mathematical induction
    \end{enumerate}
\end{enumerate}
\pagebreak
\begin{enumerate}
    \item[2.] \reversemarginpar\marginnote{ \fbox{10 Marks} }[-0.24in] Write the negation of each of the following and prove or disprove the original statement.
    \begin{enumerate}
        \item For all $x \in \mathbb{R}$, there exists $y \in \mathbb{R}$ such that for all $z \in \mathbb{R}$, if $x + y < z$, then $x -y > z$.
    \end{enumerate}
\end{enumerate}
\pagebreak
\begin{enumerate}
    \item[] \begin{enumerate}
        \item[(b)] There exists $x \in \mathbb{R}$ such that for all $y \in \mathbb{R}$, for all $z \in \mathbb{R}$, $xy > z$.
    \end{enumerate}
\end{enumerate}
\pagebreak
\begin{enumerate}
    \item[3. ]\reversemarginpar\marginnote{ \fbox{10 Marks} }[-0.24in] Let $f: A \mapsto B$ and $g: B \mapsto C$ be functions. Prove or disprove each of the following: \begin{enumerate}
        \item For all $U \subseteq C$, $(g \circ f)^{-1} (U) = f^{-1}(g^{-1}(U))$.
        
    \end{enumerate}
\end{enumerate}
\pagebreak
\begin{enumerate}
    \item[] \begin{enumerate}
        \item[(b)] For all $U \subseteq B$, $(g\circ f)^{-1}(g(U)) = f^{-1}((U))$
    \end{enumerate}
\end{enumerate}
\pagebreak
\begin{enumerate}
    \item[4.]\reversemarginpar\marginnote{ \fbox{10 Marks} }[-0.24in] Let $n \in \mathbb{N}$ be even and $\mathbb{Z} / n\mathbb{Z} = \set{0,1,2,\ldots,n-1}$. Let $S = \set{k \in \mathbb{Z} / n\mathbb{Z}| 2k \equiv 0 \mod n } \subseteq \mathbb{Z} / n\mathbb{Z}$. Prove that $|S|$ is odd.
\end{enumerate}
\pagebreak
\begin{enumerate}
    \item[5.]  \reversemarginpar\marginnote{ \fbox{10 Marks} }[-0.24in] Prove that $f: \mathbb{R} \mapsto \mathbb{C}\backslash \set{0}$ where $f(x) = e^{2\pi i x}$ is injective and for all $x,y \in \mathbb{R}$, $f(x+y) = f(x)f(y)$.
\end{enumerate}
\pagebreak
\begin{enumerate}
    \item[] \begin{enumerate}
        \item[(b)] Let $R$ be a relation on $\mathbb{R}$ be defined as $x R y$ if and only if $x = y + z$ where $z \in f^{-1}(1)$. Prove that $R$ is an equivalence relation.
    \end{enumerate}
\end{enumerate}
\pagebreak
\begin{enumerate}
    \item[] \begin{enumerate}
        \item[(c)] Find all equivalence classes under $R$. Show that the operation $[a] + [b] = [a+b]$ is well defined for $a,b \in \mathbb{R}$.
    \end{enumerate}
\end{enumerate}
\pagebreak
\begin{enumerate}
    \item[] \begin{enumerate}
        \item[(d)] Let $\mathbb{R}/R$ denotes the set of equivalence classes under $R$. Find a bijective map $g : \mathbb{R}/R \mapsto \mathrm{Im}(f)$ such that $g([x] + [y]) = g([x])g([y])$. Prove your result.
    \end{enumerate}
\end{enumerate}
\pagebreak
\begin{enumerate}
    \item[6.]  \begin{enumerate}
    \item \reversemarginpar\marginnote{ \fbox{10 Marks} }[-0.24in] 
Let $X$ be a set. Prove that any equivalence relation on $X$ forms a partition on $X$.
\end{enumerate}
\end{enumerate}
\pagebreak
\begin{enumerate}
    \item[] \begin{enumerate}
        \item[(b)] Prove that any partition on $X$ corresponds to equivalence classes of an equivalence relation on $X$.
    \end{enumerate}
\end{enumerate}
\pagebreak
\begin{enumerate}
    \item[7.] \begin{enumerate}
        \item \reversemarginpar\marginnote{ \fbox{10 Marks} }[-0.24in] Prove that if $a \neq 0$, $\lim_{x \rightarrow a} \frac{1}{x} = \frac{1}{a}$.
    \end{enumerate}
\end{enumerate}
\pagebreak
\begin{enumerate}
    \item[] \begin{enumerate}
        \item[(b)] Prove that $\lim_{x \rightarrow 0} \frac{1}{x}$ does not exist.
    \end{enumerate}
\end{enumerate}
\pagebreak
\begin{enumerate}
    \item[8.] \reversemarginpar\marginnote{ \fbox{10 Marks} }[-0.24in] Prove or disprove each of the following: \begin{enumerate}
        \item Let $A,B \subseteq C$. If $|A\backslash C| = |B \backslash C|$, then $|A| = |B|$
    \end{enumerate}
\end{enumerate}
\pagebreak
\begin{enumerate}
    \item[] \begin{enumerate}
        \item[(b)] Let $A_i \in X$. Then, $(\bigcup_{i \in I} A_i) \backslash X = \bigcap_{i \in I} (A_i \backslash X)$.
    \end{enumerate}
\end{enumerate}
\pagebreak
\begin{enumerate}
    \item[9.] \reversemarginpar\marginnote{ \fbox{10 Marks} }[-0.24in] Prove or disprove that the set of decimal numbers $0.a_1 a_2 a_3 \ldots$ where $a_1,a_2,a_3 \in \set{0,1}$ is uncountable.
\end{enumerate}
\pagebreak
\begin{enumerate}
    \item[10.] \reversemarginpar\marginnote{ \fbox{10 Marks} }[-0.24in] 
Let $N= \set{1,2,3,\ldots,n}$ for some $n \in \mathbb{N}$, and let $S_n$ be the set of bijective functions $f: N \mapsto N$. \begin{enumerate}
    \item Prove by induction that for all $n \in \mathbb{N}$, $|S_n|=n!$.
\end{enumerate}
\end{enumerate}
\pagebreak
\begin{enumerate}
   \item[] \begin{enumerate}
       \item[(b)]  Prove by induction that for all $n \geq 3 \in \mathbb{N}$, there exists $f,g \in S_n$ such that $f \circ g \neq g \circ f$.
   \end{enumerate}
\end{enumerate}
\end{document}
