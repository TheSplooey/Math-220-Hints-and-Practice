\documentclass[letterpaper,12pt]{article}
\newcommand{\myname}{Hints}
\newcommand{\mynumber}{Hints} 
\newcommand{\hw}{9} 
\usepackage{amsmath, amsfonts, amssymb, amsthm}

\usepackage[paper=letterpaper,left=25mm,right=25mm,top=3cm,bottom=25mm]{geometry}


\usepackage{fancyhdr} %% for details on how this work, search-engine ``fancyhdr documentation''
\pagestyle{fancy}

\lhead{Mathematics 220} % course name as top-left
\chead{Homework \hw} % homework number in top-centre
\rhead{ \myname \\ \mynumber }

\lfoot{\myname} % name on bottom-left
\cfoot{Page \thepage} % page in middle
\rfoot{\mynumber} % student number on bottom-right

\renewcommand{\headrulewidth}{0.4pt}
\renewcommand{\footrulewidth}{0.4pt}

\newcommand{\set}[1]{\left\{ #1 \right\}}
%% We also redfine the negation symbol:
\renewcommand{\neg}{\sim}
\newtheorem{lemma}{Lemma}[section]
\begin{document}

\subsection*{Hints to homework 9:}
\begin{enumerate}
\item[1.] See if you remember any theorems about what happens if $g \circ f$ is bijective to get an idea of whether it is true or not, and why so. When doing the actual proof I am guessing the intended solution is for you NOT to use these theorems however.
\item[3.] Try drawing it out. See the image attached.
\end{enumerate}
The rest is just checking definitions or seeing if you remember how to use some tools from high school relating to functions.
\end{document}
