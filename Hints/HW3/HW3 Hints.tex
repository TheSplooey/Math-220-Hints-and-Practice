
\documentclass[letterpaper,12pt]{article}
\newcommand{\myname}{Hints}
\newcommand{\mynumber}{Hints} 
\newcommand{\hw}{3} 
\usepackage{amsmath, amsfonts, amssymb, amsthm}

\usepackage[paper=letterpaper,left=25mm,right=25mm,top=3cm,bottom=25mm]{geometry}


\usepackage{fancyhdr} %% for details on how this work, search-engine ``fancyhdr documentation''
\pagestyle{fancy}

\lhead{Mathematics 220} % course name as top-left
\chead{Homework \hw} % homework number in top-centre
\rhead{ \myname \\ \mynumber }

\lfoot{\myname} % name on bottom-left
\cfoot{Page \thepage} % page in middle
\rfoot{\mynumber} % student number on bottom-right

\renewcommand{\headrulewidth}{0.4pt}
\renewcommand{\footrulewidth}{0.4pt}

\newcommand{\set}[1]{\left\{ #1 \right\}}
%% We also redfine the negation symbol:
\renewcommand{\neg}{\sim}
\newtheorem{lemma}{Lemma}[section]
\begin{document}

\subsection*{Hints to homework 3:}
\begin{enumerate}
\item[1.] Just be careful that when negating quantifiers you have to switch "for all" and "there exists" and you do not change what comes after the quantifier. Also make sure to negate the inequality correctly.
\item[2.] Notice if the hypothesis is false, then the statement is always true.
\item[3.] For (c), consider what happens when both 3 and 4 divide $n$. For (d), consider addition mod 3 and mod 4.
\end{enumerate}
The rest are pretty straight forward.
\end{document}