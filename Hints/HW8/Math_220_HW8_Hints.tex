\documentclass[letterpaper,12pt]{article}
\newcommand{\myname}{Hints}
\newcommand{\mynumber}{Hints} 
\newcommand{\hw}{8} 
\usepackage{amsmath, amsfonts, amssymb, amsthm}

\usepackage[paper=letterpaper,left=25mm,right=25mm,top=3cm,bottom=25mm]{geometry}


\usepackage{fancyhdr} %% for details on how this work, search-engine ``fancyhdr documentation''
\pagestyle{fancy}

\lhead{Mathematics 220} % course name as top-left
\chead{Homework \hw} % homework number in top-centre
\rhead{ \myname \\ \mynumber }

\lfoot{\myname} % name on bottom-left
\cfoot{Page \thepage} % page in middle
\rfoot{\mynumber} % student number on bottom-right

\renewcommand{\headrulewidth}{0.4pt}
\renewcommand{\footrulewidth}{0.4pt}

\newcommand{\set}[1]{\left\{ #1 \right\}}
%% We also redfine the negation symbol:
\renewcommand{\neg}{\sim}
\newtheorem{lemma}{Lemma}[section]
\begin{document}

\subsection*{Hints to homework 8:}
\begin{enumerate}
\item[4.] See the image attached in the folder.
\item[5.] Be very careful that the relation is defined on $\mathcal{P}(E)$. 
\item[6.] Be careful, by $[u]_n$ "has a multiplicative inverse" it means that there exists $[v]_n$ such that $[u]_n\cdot [v]_n = [1]_n$ with multiplication being defined based on the usual modular arithmetic. It is interesting, but not helpful, to note that in this problem, you are partitioning $\mathbb{Z}/n\mathbb{Z}$ based on generators for distinct subgroups of $\mathbb{Z}/n\mathbb{Z}$ under addition where generators lie in the same equivalence class if they generate the same subgroup.
\end{enumerate}
A lot of this is just definition checking. You may find it useful to split some of the questions into cases.
\end{document}
