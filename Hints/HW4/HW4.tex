\documentclass[letterpaper,12pt]{article}
\newcommand{\myname}{Hints}
\newcommand{\mynumber}{Hints} 
\newcommand{\hw}{4} 
\usepackage{amsmath, amsfonts, amssymb, amsthm}

\usepackage[paper=letterpaper,left=25mm,right=25mm,top=3cm,bottom=25mm]{geometry}


\usepackage{fancyhdr} %% for details on how this work, search-engine ``fancyhdr documentation''
\pagestyle{fancy}

\lhead{Mathematics 220} % course name as top-left
\chead{Homework \hw} % homework number in top-centre
\rhead{ \myname \\ \mynumber }

\lfoot{\myname} % name on bottom-left
\cfoot{Page \thepage} % page in middle
\rfoot{\mynumber} % student number on bottom-right

\renewcommand{\headrulewidth}{0.4pt}
\renewcommand{\footrulewidth}{0.4pt}

\newcommand{\set}[1]{\left\{ #1 \right\}}
%% We also redfine the negation symbol:
\renewcommand{\neg}{\sim}
\newtheorem{lemma}{Lemma}[section]
\begin{document}

\subsection*{Hints to homework 4:}
\begin{enumerate}
\item[1.] You know from Bézout's lemma that there exists $x,y \in \mathbb{Z}$ such that $ax + by = 1$, so you can fix these variables. Try playing around with expressions by fixing $x$ and $y$.
\item[3.] Usually, multiply things out and simplifying expressions like $|\frac{2x^2}{2x^2 + 1}-2|$ will make things less confusing. 
\item[4.] You should think about what special properties sine have. It would be really nice if I can just somehow "delete" sine, or just not have to deal with it at all.
\item[5.] Since $x_n$ converges to 0, you can fix any arbitrary $\varepsilon_N > 0$ and you will get that for all $n > N$, $|x_n - 0 | \leq \varepsilon_N$. Notice $\set{x_1,x_2,x_3,\ldots,x_N}$ is a finite set.
\item[6.] What special properties do $(-1)^n$ have? What happens when $n$ is very big?
\end{enumerate}
(2) is pretty straight forward, just be careful.
\end{document}
