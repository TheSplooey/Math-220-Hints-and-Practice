\documentclass[letterpaper,12pt]{article}
\newcommand{\myname}{Hints}
\newcommand{\mynumber}{Hints} 
\newcommand{\hw}{7} 
\usepackage{amsmath, amsfonts, amssymb, amsthm}

\usepackage[paper=letterpaper,left=25mm,right=25mm,top=3cm,bottom=25mm]{geometry}


\usepackage{fancyhdr} %% for details on how this work, search-engine ``fancyhdr documentation''
\pagestyle{fancy}

\lhead{Mathematics 220} % course name as top-left
\chead{Homework \hw} % homework number in top-centre
\rhead{ \myname \\ \mynumber }

\lfoot{\myname} % name on bottom-left
\cfoot{Page \thepage} % page in middle
\rfoot{\mynumber} % student number on bottom-right

\renewcommand{\headrulewidth}{0.4pt}
\renewcommand{\footrulewidth}{0.4pt}

\newcommand{\set}[1]{\left\{ #1 \right\}}
%% We also redfine the negation symbol:
\renewcommand{\neg}{\sim}
\newtheorem{lemma}{Lemma}[section]
\begin{document}

\subsection*{Hints to homework 7:}
\begin{enumerate}
\item[2.] You may want to try playing around with logical equivalences of the statements (for example, contrapositive). Drawing pictures might help you visualize too.
\item[6.] What equivalence class does $e$ belong to and how would it look like? What about $e^2$ and so on? What elements do these equivalence classes have in common and not havei n common?
\end{enumerate}
1,3,4,5 are pretty straightforward, play around with the definitions a bit, and for Q1 you may want to write out the elements of $\mathcal{P}(\set{1,2})$. Do not be alarmed however if you are stuck, this material is more abstract than what you have dealt with in the past so it can be confusing, but once you understand the definitions and what you need to prove, it will get a lot easier. An additional note would be that for proofs involving symmetry and transitivity, it suffices to prove $aRb$ implies $bRa$ since the choice of $a$ and $b$ are arbitrary at the beginning of your assumption (You start the proof with "Let $a,b \in A$"). 
\end{document}
